\documentclass[a4paper,10pt]{book}

\usepackage{ifxetex,ifluatex}

% IDIOMA

\ifxetex
  \usepackage{polyglossia}
  \setmainlanguage{spanish}
  % Fuente
  \usepackage{fontspec}
  \setmainfont{DejaVu Serif}
  
  % Tabla en lugar de cuadro
  \gappto\captionsspanish{
  \renewcommand{\tablename}{Tabla}%
  \renewcommand{\listtablename}{Índice de tablas}%
  }
  
\else
  \usepackage[spanish,es-tabla]{babel}
  \usepackage[utf8]{inputenc} 
  \usepackage[T1]{fontenc}
  \usepackage{DejaVuSerif}
\fi

\usepackage{amsmath, amsthm, amssymb}

% Mejora el aspecto del documento
\usepackage{microtype}
\UseMicrotypeSet[protrusion]{basicmath} % disable protrusion for tt fonts

\usepackage[usenames,dvipsnames,svgnames,table]{xcolor}

% Hipervínculos

\ifxetex
  \usepackage[setpagesize=false, % page size defined by xetex
              unicode=false, % unicode breaks when used with xetex
              xetex]{hyperref}
\else
  \usepackage[unicode=true]{hyperref}
\fi

\hypersetup{unicode=true,
            colorlinks=true,
            linkcolor=Black,
            citecolor=Green,
            urlcolor=Blue,
            breaklinks=true}
\urlstyle{same}  % don't use monospace font for urls

\usepackage{listings}

\lstset{
	tabsize=2,
	basicstyle=\small\ttfamily,
        breaklines=true,
        columns=fixed,
        extendedchars=true,
        prebreak = \raisebox{0ex}[0ex][0ex]{\ensuremath{\hookleftarrow}},
        tabsize=2,
        backgroundcolor=\color[HTML]{F0F0F0},
        identifierstyle=\ttfamily\color[HTML]{06287E},
        keywordstyle=\bfseries\color[HTML]{007020},
        commentstyle=\itshape\color[HTML]{60A0B0},
        stringstyle=\color[HTML]{4070A0},      
	}

% Listings no acepta UTF8        
\lstset{literate=%
{á}{{\'a}}1
{é}{{\'e}}1
{í}{{\'i}}1
{ó}{{\'o}}1
{ú}{{\'u}}1
{Á}{{\'A}}1
{É}{{\'E}}1
{Í}{{\'I}}1
{Ó}{{\'O}}1
{Ú}{{\'U}}1
{ñ}{{\~n}}1
{ü}{{\"u}}1
{Ü}{{\"U}}1
}

\addto\captionsspanish{\renewcommand{\lstlistingname}{Código}}

\usepackage{longtable,booktabs}

\usepackage{graphicx,grffile}

\usepackage{parskip}

% Márgenes
\usepackage[top=1in,
            bottom=1in,
            left=1.8in,
            right=1.2in,
            headheight=30pt,
            includeheadfoot=true]{geometry}


\title{Curso no convencional de \LaTeX}

\author{Ondiz Zarraga}

\date{\today}

\begin{document}

\frontmatter
\maketitle

\newpage % Página en blanco
 
\null % Porque \vfill necesita algo delante para funcionar
\vfill % Para que lo escriba al final de la página
 
\includegraphics[width=0.3\textwidth]{docs/Figuras/by-sa}\\
Este trabajo se comparte bajo una licencia
\href{https://creativecommons.org/licenses/by-sa/4.0/}
{Creative Commons Atribución - Compartir igual}.


\thispagestyle{empty}

\tableofcontents

%\listoftables
%\listoffigures

\chapter*{Prefacio}
\chaptermark{Prefacio} % marcar como capítulo para que cambie encabezado
\addcontentsline{toc}{chapter}{Prefacio}
\input{Contenido/Prefacio}

\mainmatter

\chapter{Introducción}
\input{Contenido/01.Introduccion}

\chapter{¿Qué necesito?}
\input{Contenido/02.QueNecesito}

\chapter{Un documento básico}
En este capítulo vamos a crear nuestro primer documento, muy
emocionante todo. Pero antes tenemos que hablar de estructura,
entornos y sintaxis. ¡Ánimo!

\section{Un documento y sus partes}\label{un-documento-y-sus-partes}

Un documento escrito en LaTeX tiene esta pinta\footnote{Este ejemplo
  está adaptado del
  \href{https://github.com/ekaitz-zarraga/programming-notes}{repo de
  apuntes de programación} de mi señor hermano y mío. Es una versión
  simplificada.}:

\begin{lstlisting}[language={[latex]tex}, caption={Ejemplo de documento escrito con \LaTeX}]

\documentclass[a4paper,10pt]{article}

% PREÁMBULO 

% Paquetes

\usepackage[utf8]{inputenc}
\usepackage[spanish,es-tabla]{babel}
\usepackage[T1]{fontenc}

\usepackage{listings}

% Comandos

\renewcommand{\lstlistingname}{Código}
\renewcommand{\lstlistlistingname}{Índice de fragmentos de código fuente}

% Opciones

\title{Python 2.*}
\author{Ondiz Zarraga}

%%%%%%%%%%%%%%%%%%%%%%%

\begin{lstlisting}[language={[latex]tex}]

\begin{document}
\maketitle

\begin{abstract}
Este documento es una pequeña guía de Python 
\end{abstract}

\tableofcontents

\section{Sobre el lenguaje}

\begin{itemize}
    \item Interpretado
    \item Indentación obligatoria
    \item Distingue mayúsculas - minúsculas
    \item No hay declaración de variables (\textit{dynamic typing})
    \item Orientado a objetos  
    \item Garbage colector: quita los objetos a los que no haga referencia nada
\end{itemize}

\end{document}
\end{lstlisting}

El documento tiene dos cosas llamativas:

\begin{itemize}
\item
  Cosas que empiezan por contrabarra, los \textbf{comandos}
\item
  Cachos que van entre un \lstinline!\begin! y un \lstinline!\end!, los
  \textbf{entornos}
\end{itemize}

Los comandos son la manera en la que nos comunicamos con LaTeX y un
entorno es el pedazo donde se aplica un formato.

Un entorno superimportante es el entorno \lstinline!document!, ahí
dentro irá el contenido de nuestro documento. Todo lo que va entre la
definición de documento (\lstinline!\documentclass!) y el inicio del
entorno \lstinline!document! se conoce como \textbf{preámbulo} y es
donde se cargan paquetes (\lstinline!\usepackage!), se definen comandos
y se establecen opciones.

\section{Nuestro primer documento}\label{nuestro-primer-documento}

Ahora que sabemos cómo es un documento vamos a empezar a escribir uno.
El documento más básico que podemos hacer es este:

\begin{lstlisting}[language={[latex]tex}]
\documentclass{article}

\begin{document}
Hola
\end{document}
\end{lstlisting}

Esto podemos compilarlo con el botoncillo de compilar de nuestro IDE, en
la terminar o con Pandoc.

Para compilar en la terminal haríamos (imaginemos que nuestro documento
se llama \lstinline!hola.tex!):

\begin{lstlisting}[language=bash]
pdflatex hola.tex
\end{lstlisting}

Evidentemente podéis compilar también con \lstinline!xelatex! de manera
equivalente. Si quisiéramos utilizar Pandoc tendríamos que hacer lo
siguiente:

\begin{lstlisting}[language=bash]
pandoc hola.tex -o hola.pdf
\end{lstlisting}

En los tres casos el resultado es el mismo excepto porque Pandoc no nos
deja
\href{https://ondahostil.wordpress.com/2016/11/17/lo-que-he-aprendido-archivos-auxiliares-de-latex/}{archivos
auxiliares} por ahí bailando.

¡Ya hemos escrito un documento! Debo reconocer que no es un documento
demasiado interesante, para hacer algo más chulo tenemos que aprender un
poco más de sintaxis.

\section{Un poco de sintaxis}\label{un-poco-de-sintaxis}

Para poder escribir un documento un poco más interesante tenemos que
aprendernos unos pocos comandos, hoy os voy a hablar de algunos
sencillos para que le vayáis cogiendo el truco, en las siguientes
entregas entraremos más en detalle.

Antes de nada una cosa, aun no hemos aprendido a establecer las opciones
de idioma y por lo tanto tendremos problemas en los idiomas con acentos.
Si queréis poneros a jugar ya, ponedme esto en el preámbulo, justo
debajo de \lstinline!\documentclass!:

\begin{lstlisting}[language={[latex]tex}]
\usepackage[spanish]{babel} % sustituir spanish por el idioma
\usepackage[utf8]{inputenc}
\usepackage[T1]{fontenc}
\usepackage{lmodern}
\end{lstlisting}

En el Capítulo \ref{ch:idioma} analizaremos mejor este tema.

\subsection{Título, capítulos y
secciones}\label{tuxedtulo-capuxedtulos-y-secciones}

Una cosa importante de LaTeX es que nos desacopla el contenido del
documento de su formato. Con esto quiero decir que nosotros le diremos
cuál es el título del documento y dónde comienzan las secciones y él les
dará el formato correspondiente según el tipo de documento y las
opciones que hayamos establecido previamente.

Pongamos por ejemplo los tipos de documento \lstinline!article! y
\lstinline!book!. Como su nombre indica, el primero se utiliza para
escribir artículos y el segundo libros. Como LaTeX es muy listo, cuando
le digamos que escriba el título, para el caso del artículo nos lo
escribirá en la parte superior de la página con el texto debajo, pero
para el caso del libro nos creará una portada. Para ambos casos la
sintaxis es exactamente la misma:

\begin{lstlisting}[language={[latex]tex}]
\documentclass{article} % O book

% Definimos el título
\title{Título del documento}

\begin{document}
\maketitle % Creamos el título
\end{document}
\end{lstlisting}

Del mismo modo, nosotros solo le tenemos que decir el título de la
sección o el capítulo y él le dará el formato correspondiente. Otra
diferencia entre las clases \lstinline!article! y \lstinline!book! es
que \lstinline!article! no tiene capítulos, como es lógico.

Para definir capítulos y secciones utilizamos los comandos
\lstinline!\chapter! y \lstinline!\section! en el cuerpo del documento,
es decir, después de \lstinline!\begin{document}!. Por ejemplo:

\begin{lstlisting}[language={[latex]tex}]
\chapter{Capítulo numerado}
\section{Primera sección}
\subsection{Primera subsección}

\chapter*{Capítulo sin numerar}
\section*{Sección sin numerar}
\end{lstlisting}

Como veis, podemos usar los comandos de sección y capítulo con el
asterisco para que no nos las numere. Otra cosa interesante es el grado
de anidación, tenemos secciones, subsecciones, subsubsecciones, párrafos
(\lstinline!\paragraph!) y subpárrafos (\textbackslash{}subparagraph),
cada uno con su formato definido. La clase \lstinline!book!\footnote{La
  clase \lstinline!book! no es la única que tiene partes y tal, pero de
  momento así nos vale.} también tiene por encima de las secciones,
capítulos y partes (\lstinline!\part!). Más adelante aprenderemos a
personalizar todos estos formatos porque si hay algo bueno que tiene
LaTeX es que nos deja cambiar \emph{absolutamente todo} y con relativa
facilidad (gracias a StackExchange, especialmente).

\subsection{Listas}\label{listas}

Si sois como yo os gustará especialmente este apartado: las listas.
LaTeX tiene dos tipos de listas: las numeradas y las sin numerar. Son
respectivamente los entornos \lstinline!enumerate! e
\lstinline!itemize!. Se usan exactamente igual, así que solo pongo el
ejemplo de uno de ellos:

\begin{lstlisting}[language={[latex]tex}]
\begin{itemize}
  \item Primer ítem
  \item Segundo ítem
\end{itemize}
\end{lstlisting}

Lo mejor es que podemos mezclar y anidar estos dos entornos que el
simbolito y la indentación cambiarán solos sin que nos tengamos que
preocupar. Por ejemplo,

\begin{lstlisting}[language={[latex]tex}]
\begin{enumerate}
    \item Primer ítem
    \item Segundo ítem con subítems
    \begin{itemize}
        \item Ítem sin numerar
    \end{itemize}
    \item Tercer ítem
\end{enumerate}
\end{lstlisting}

Quedará así:

 \begin{center}\rule{0.5\textwidth}{0.1mm}\end{center}


\begin{enumerate}
\item
  Primer ítem
\item
  Segundo ítem con subítems

  \begin{itemize}
  \item
    Ítem sin numerar
  \end{itemize}
\item
  Tercer ítem
\end{enumerate}

\begin{center}\rule{0.5\textwidth}{0.1mm}\end{center}

\subsection{Imágenes}\label{imuxe1genes}

Hoy solo vamos a ver como colocar una única imagen, que lo de las
imágenes tiene un poco de lío. Lo que debemos saber es lo siguiente:

\begin{itemize}
\item
  Necesitamos llamar al paquete \lstinline!graphicx! en el preámbulo.
  Para eso escribiremos \lstinline!\usepackage{graphicx}! entre
  \lstinline!\documentclass! y 
    \lstinline!\begin{document}!
\item
  Las imágenes se insertan con el comando
  \lstinline!\includegraphics[opciones]{ruta}!
\item
  Si queremos ponerles un pie de figura y una etiqueta, decidir su
  posición en la página y demás necesitamos el entorno
  \lstinline!figure!
\end{itemize}

Aquí tenemos un ejemplo de cómo insertar una imagen con en el entorno
\lstinline!figure!:

\begin{lstlisting}[language={[latex]tex}]
\begin{figure}[h] % opción de posicionamiento
    \caption{Pie de imagen}
    \centering % imagen centrada
    % Imagen 50% de ancho del texto
    \includegraphics[width=0.5\textwidth]{ruta_a_la_imagen}
\end{figure}
\end{lstlisting}

Una cosa importante de LaTeX son los objetos \emph{flotantes}, es decir,
los que si no les obligamos, se colocan en el hueco que mejor les venga
del documento. Esto es lo que nos ocurre con las imágenes al usar el
entorno \lstinline!figure!. Nos ayuda a que no haya huecos chungos en el
documento pero a veces junta todas las imágenes en una misma página o al
final del documento. Para evitar esto tenemos las opciones de
posicionamiento, de las que hablaremos cuando profundicemos en las
imágenes.

\subsection{Tablas}\label{tablas}

Para mí las tablas son lo peor de todo LaTeX. Son la cosa menos amigable
que se puede echar uno a la cara. Con un IDE la cosa mejora, pero
imaginaros cómo será el tema que la mitad de las veces las creo online
\href{http://www.tablesgenerator.com/}{aquí} y luego pego el resultado.

Lo que debemos de saber de las tablas es lo siguiente:

\begin{itemize}
\item
  El contenido se especifica con el entorno \lstinline!tabular!.
  Separemos las celdas con el ampersand y cambiaremos de línea con dos
  contrabarras.
\item
  Si queremos ponerles un pie de tabla y una etiqueta, decidir su
  posición en la página y demás necesitamos el entorno \lstinline!table!
\item
  Si utilizamos el entorno \lstinline!table! la tabla se volverá
  \emph{flotante}
\end{itemize}

Aquí tenemos un ejemplo de tabla sencilla:

\begin{lstlisting}[language={[latex]tex}]
\begin{table}
    \begin{tabular}{|ll|}
      \hline % Separador
      Columna 1 & Columna 2 \\
      1         & 2         \\
      3         & 4         \\
      \hline
    \end{tabular}
    \caption {Pie de tabla}
\end{table}
\end{lstlisting}

Al igual que con la imágenes, profundizaremos en las tablas más
adelante.

\subsection{Ecuaciones}\label{ecuaciones}

Las ecuaciones son la razón por la que mucha gente se pasa a LaTeX. Son
muy cucas y escribirlas, una vez cogido el callo, no es un infierno (de
nuevo, ¡hola, Word!). Pero no nos vamos a engañar, al principio es
\emph{muerte y destrucción}. Hay dos tipos de ecuaciones en LaTeX: las
que van dentro de la línea y las que tienen línea propia. Las primeras
van entre signos de dólar y las segundas dentro del entorno
\lstinline!equation!. Aquí tenemos un ejemplo:

\begin{lstlisting}[language={[latex]tex}]
Imaginemos que $a=1$ en la siguiente ecuación:
\begin{equation}
ax^2 + 1 = 0
\end{equation}
\end{lstlisting}

Del mismo modo que ocurría con las secciones, si utilizamos el asterisco
la ecuación no estará numerada.

El lío con las ecuaciones es que todo se define con comandos, por
ejemplo, \lstinline!\frac{numerador}{denominador}! se usa para escribir
una fracción y \lstinline!\omega! para la letra griega omega. Los que
uséis un IDE lo tenéis más fácil porque suelen tener una barrita con los
símbolos más usados, a los demás les tocará investigar.

También tenemos otras herramientas que nos pueden ayudar a escribir
ecuaciones:

\begin{itemize}
\item
  \textbf{Editores online}: son editores con su GUI y tal que nos
  facilitan la tarea cuando todavía no conocemos los comandos de LaTeX,
  por ejemplo tenemos
  \href{http://www.numberempire.com/texequationeditor/equationeditor.php\%22}{Latex
  Equation Editor} o
  \href{https://www.latex4technics.com/}{LaTeX4technics}.
\item
  \href{http://detexify.kirelabs.org/classify.html}{\textbf{Detexify}}:
  un cacharro que nos busca el símbolo de LaTeX más parecido a algo que
  le pintemos.
\end{itemize}

Las ecuaciones se merecen una entrada propia y la tendrán.

\section{Bonus: opción Markdown}\label{bonus-opciuxf3n-markdown}

Si tenemos instalado Pandoc todo lo que hemos explicado aquí podemos
conseguirlo escribiendo en Markdown
(\href{http://rmarkdown.rstudio.com/authoring_pandoc_markdown.html}{sabor
Pandoc}) y convirtiendo a pdf. Por ejemplo, podemos meter una imagen
como:

\begin{lstlisting}
![Pie de imagen](/ruta){width=0.7 #etiqueta}
\end{lstlisting}

\section{Referencias}\label{referencias}

\href{https://en.wikibooks.org/wiki/LaTeX/Document_Structure}{\emph{LaTeX/Document
Structure} en Wikibooks}

\href{https://www.sharelatex.com/learn/Environments}{\emph{Environments}
en ShareLaTeX}

\href{http://www.personal.ceu.hu/tex/environ.htm}{\emph{Latex Standard
Environments}}

\href{http://texblog.org/2013/02/13/latex-documentclass-options-illustrated/}{\emph{LaTeX
documentclass options illustrated}}

\href{https://www.sharelatex.com/learn/Sections_and_chapters}{\emph{Sections
and chapters} en ShareLaTeX}

\href{https://en.wikibooks.org/wiki/LaTeX/Tables}{\emph{LaTeX/Tables} en
Wikibooks}

\href{https://www.sharelatex.com/learn/Inserting_Images}{\emph{Inserting
images} en ShareLaTeX}

\href{http://osl.ugr.es/CTAN/info/symbols/comprehensive/symbols-a4.pdf}{\emph{The
Comprehensive LaTeX Symbol List}}

\href{ftp://ftp.ams.org/pub/tex/doc/amsmath/short-math-guide.pdf}{\emph{Short
Math Guide for LaTeX}}


\chapter{Insertando figuras}
\input{Contenido/04.Figuras}

\chapter{LaTeX y las ecuaciones}
Siguiendo con el desmenuce de la sintaxis, vamos a hablar de ecuaciones,
el motivo por el que mucha gente (\emph{científicos}) se pasan a LaTeX.
Como dijimos el otro día, hay dos tipos de ecuaciones en LaTeX:

\begin{itemize}
\item
  las que van \textbf{dentro de una línea}, que se escriben entre signos
  de dólar y se suelen conocer como \emph{inline}
\item
  las que tienen \textbf{línea propia}, que usan el entorno
  \lstinline!equation! o el atajo\footnote{Yo no suelo usar el atajo
    porque me resulta más difícil de leer, pero, oyes, para gustos
    colores.} \lstinline!\[...\]!
\end{itemize}

Aquí tenemos un ejemplo usando los dos tipos:

\begin{lstlisting}[language={[latex]tex}]
\begin{equation}

  e^{i\pi} + 1 = 0

\end{equation}

donde $i =\sqrt{-1}$
\end{lstlisting}

Como veis, escribimos las ecuaciones mediante comandos, algo que
inicialmente parece un atraso pero que cuando cogemos un poco de
práctica, es terriblemente eficaz. Si estáis usando un editor
específico, tendréis una barra con los símbolos más usados, es una buena
forma de empezar con las ecuaciones. Más abajo os hablo de la sintaxis
más en detalle y doy unos ejemplos. Al escribir ecuaciones es
recomendable cargar los siguientes paquetes:

\begin{itemize}
\item
  \href{https://www.ctan.org/pkg/amsmath}{\lstinline!amsmath!} (AMS
  Math), que mejora el comportamiento y el aspecto de las ecuaciones.
  Nos permite, por ejemplo, añadir un asterisco en el entorno
  \lstinline!equation! para crear ecuaciones sin numerar.
\item
  \href{https://www.ctan.org/pkg/amsthm}{\lstinline!amsthm!} (AMS
  Theorem), que define los entornos teorema y demostración.
\item
  \lstinline!amssymb! (AMS Symbol), que carga a su vez
  \lstinline!amsfonts! e incluye una colección de símbolos matemáticos.
\end{itemize}

Podemos cargarlos todos a la vez añadiendo esta línea al
\emph{preámbulo}\footnote{Recordemos: el \emph{preámbulo} es lo que hay entre la
definición del documento (\lstinline!\\documentclass!) y el inicio del entorno
\lstinline!document! (\lstinline!\\begin\{document\}!).}:

\begin{lstlisting}[language={[latex]tex}]
\usepackage{amsmath, amsthm, amssymb}
\end{lstlisting}

Ese AMS que precede a todos ellos viene de
\href{http://www.ams.org/home/page}{\emph{American Mathematical
Society}}, los que originalmente desarrollaron estos paquetes.

\section{Comandos}\label{sec:comandos}

Vamos a ver un poco de sintaxis, pero antes de nada os dejo un par de
herramientas interesantes sobre todo para los novatillos (o
\emph{Nóbeles} que decía mi profe de autoescuela, \emph{conductor Nóbel}
(sic)):

\begin{itemize}
\item
  \textbf{Editores de ecuaciones online}: hasta que le vayamos cogiendo
  el callo a las ecuaciones, aparte de la barrita del IDE tenemos
  editores online como
  \href{http://www.numberempire.com/texequationeditor/equationeditor.php}{este}
  o
  \href{http://www.numberempire.com/texequationeditor/equationeditor.php}{este
  otro} que es más cuco.
\item
  \textbf{Detexify}: si no sabemos cómo se llama un símbolo y, por lo
  tanto, no podemos buscar su comando tenemos
  \href{http://detexify.kirelabs.org/classify.html}{Detexify}, un
  cacharro en el que pintamos el símbolo que estamos buscando y nos
  localiza los más parecidos. Especialmente útil con la típica duda de
  \emph{¿esa letra es xi o chi?} o mi favorita \emph{¿cómo se llama la R
  esa gorda de los números reales?}. Hacemos el dibujillo y hala.
\end{itemize}

\subsection{Símbolos comunes}\label{sec:simbolos}

Símbolos hay a pilas, os voy a poner unos pocos comandos aquí pero lo
mejor es que hurguéis.

\begin{itemize}
\item
  \emph{Sumas, restas y exponenciales}: se hacen con el símbolo de toda
  la vida \lstinline!+!, \lstinline!-! y \lstinline!^!
\item
  \emph{Multiplicaciones}: aquí hay variedad según los gustos, si
  queremos el punto usamos el comando \lstinline!\cdot! si nos gusta más
  el aspa usamos \lstinline!\times!. Hacedme un favor y no me uséis ni
  la equis ni el asterisco.
\item
  \emph{Raíces}: se hacen con el comando \lstinline!\sqrt{argumento}! si
  son raíces cuadradas y añadiendo el numerito como argumento opcional
  (es decir, entre corchetes) para cualquier otra
  \lstinline!\sqrt[raíz]{argumento}!
\item
  \emph{Integrales}: funcionan con el comando \lstinline!\int!, si
  queremos que tengan límites definidos no tenemos más que escribir
  \lstinline$\int_{inferior}^{superior}$. Por ejemplo, esta integral
  impropia \(\int_{0}^{\infty}\) se conseguiría así
  \lstinline$\int_{0}^{\infty}$. Si os fijáis las integrales, a
  diferencia de las raíces, no llevan llaves. Esto ocurre porque la raíz
  necesita saber cómo de largo es el contenido, la integral es
  simplemente el chirimbolo.
\item
  \emph{Sumatorios}: son como las integrales pero con el comando
  \lstinline!\sum!
\item
  \emph{Fraciones}: tan sencillas como
  \lstinline!\frac{numerador}{denominador}!
\end{itemize}

Tenéis en las referencias listas de símbolos para que les echéis una
ojeada si os parece.

\subsection{Letras griegas}\label{sec:letrasGriegas}

Una de las mejores cosas de LaTeX en mi opinión es su método para
escribir letras griegas, tan sencillo como escribir su nombre en
minúsculas para la letra en minúscula y ponerle la primera en mayúscula
para una letra griega en mayúscula. Se entenderá mejor con un ejemplo:

\begin{quote}
\lstinline!\omega! crea $\omega$ (\emph{omega minúscula}) y \lstinline!\Omega!
crea a su vez $\Omega$ (\emph{omega mayúscula})
\end{quote}

\subsection{Operadores}\label{sec:operadores}

Los operadores son las funciones cuyo nombre se escribe en texto, como
\emph{sin} o \emph{ln}. LaTeX tiene algunos de ellos definidos y es
importante usarlos para que las ecuaciones nos queden bien. Va un
ejemplo:

\begin{lstlisting}[language={[latex]tex}]
\sin^2 x + \cos^2 x = 1
\end{lstlisting}

Que crea: 

\begin{equation*}
\sin^2 x + \cos^2 x = 1 
\end{equation*}

\subsection{Matrices}\label{sec:matrices}

Funcionan de manera similar a las tablas (las columnas se separan con el
ampersand y se salta de línea con \lstinline!\\!), pero usan el entorno
\lstinline!matrix! y relacionados. El entorno \lstinline!matrix! nos
crea una matriz sin delimitadores, tendríamos que añadírselos nosotros.
Los entornos \lstinline!pmatrix!, \lstinline!vmatrix!,
\lstinline!Vmatrix! \lstinline!bmatrix! y \lstinline!Bmatrix! nos añaden
respectivamente paréntesis, barras\footnote{Como las de un determinante},
barras\footnote{Como las de una norma} dobles, corchetes y llaves. Estos
entornos que cito centran el contenido, si quisiéramos cambiar la
alineación tendríamos que usar su variantes con asterisco y darle un
argumento. Este sería un ejemplo de una matriz sencilla:

\begin{lstlisting}[language={[latex]tex}]
\begin{equation}
  \begin{matrix}
    a & b & c \\
    d & e & f \\
    g & h & i \\
  \end{matrix}
\end{equation}
\end{lstlisting}

\subsubsection{Sobre los paréntesis}\label{sec:parentesis}

Si no os apetece (como a mí) memorizar que el \lstinline!pmatrix! pone
un paréntesis y el \lstinline!vmatrix! no sé qué, podéis poner los
delimitadores vosotros según os parezca y usar siempre el entorno
\lstinline!matrix! (es lo que yo hago) pero hay que tener en cuenta una
cosa, en LaTeX hay dos tipos de paréntesis: los de tamaño fijo y los que
se adaptan al contenido. Los de tamaño fijo son tal cual el símbolo
según le damos en el teclado, los adaptativos son comandos formados por
\lstinline!\left! o \lstinline!\right!, según el lado, más el símbolo.
Por ejemplo, \lstinline!\left(! nos crea el paréntesis adaptativo del
lado izquierdo, \lstinline!\right]! el corchete adaptativo de la derecha
y así con todos. Los únicos un poco diferentes son los comandos para las
llaves, que requieren una barra de escape y son respectivamente
\lstinline!\left\{! y \lstinline!\right\}! Por ejemplo, para rodear la
matriz anterior con corchetes tendríamos que hacer lo siguiente:

\begin{lstlisting}[language={[latex]tex}]
\begin{equation}
  \left[
  \begin{matrix}
    a & b & c \\
    d & e & f \\
    g & h & i \\
  \end{matrix}
  \right]
\end{equation}
\end{lstlisting}

\section{Gestión del espacio}\label{sec:espacio}

Al igual que con el texto, LaTeX nos gestiona el espacio entre los
símbolos él solito. En general lo mejor es dejarle hacer, pero hay en
ocasiones en hay cosas que quedan \emph{feas} y hay que tocarlas un
poquito a mano. Los nazis del LaTeX nos dirán que no hay que hacer estas
cosas, que las decisiones de LaTeX deben ser respetadas. Yo no estoy de
acuerdo, la cuestión es que las ecuaciones queden a nuestro gusto. Para
ello utilizo estos dos chismes, aunque hay muchos más, que no son
específicos de las ecuaciones pero es donde suelen resultar más
necesarios:

\begin{itemize}
\item
  \lstinline!\,!: nos genera un espacio en blanco estrecho
\item
  \lstinline!~!: nos crea un
  \href{https://es.wikipedia.org/wiki/Espacio_duro}{\emph{espacio
  duro}}, es decir, un espacio que impide que se salte de línea en
  medio.
\end{itemize}

Como tampoco soy una sabia de la tipografía con estos dos me apaño, en
las referencias tenéis más y mejores explicaciones si os va el tema.

\section{Referencias cruzadas}\label{sec:refCruzadas}

Igual que las imágenes, las ecuaciones también se pueden referenciar
haciendo uso de los comandos \lstinline!\label! y \lstinline!\ref!. El
primero de ellos nos permite darle un nombre identificativo a una
ecuación y el segundo nos la cita. Al igual que ocurría con las figuras,
para poder añadir una etiqueta a una ecuación es necesario utilizar el
entorno \lstinline!equation!, no nos vale para las ecuaciones
\emph{inline}. Veamos cómo citaríamos la ecuación del primer ejemplo:

\begin{lstlisting}[language={[latex]tex}]
\begin{equation}
  e^{i\pi} + 1 = 0
  \label{eq:euler}
\end{equation}

Como vemos en la Ecuación \ref{eq:euler}
\end{lstlisting}

Que nos daría este resultado:

\begin{equation*} 
  e^{i\pi} + 1 = 0
\end{equation*}

Como vemos en la Ecuación 1

Añadir \lstinline!eq:! a la etiqueta no es necesario pero nos facilita
el trabajo al no tener las etiquetas para las figuras, las secciones y
demás elementos mezclados. También podemos definir un comando para que
nos añada la palabra \emph{Ecuación} al número. Os voy a decir cómo lo
haríamos aunque todavía no sepamos crear comandos para que veáis que es
sencillito\footnote{Creo que hay una manera mejor de hacer de definir
  este comando pero no me acuerdo y soy completamente incapaz de
  encontrarlo.}:

\begin{lstlisting}[language={[latex]tex}]
% Estructura \newcommand{\nombre}[num-args]{Descripción}
\newcommand{\refeq}[1]{Ecuación~\ref{#1}}
\end{lstlisting}

Esto mismo lo consigue el comando
\href{https://en.wikibooks.org/wiki/LaTeX/Labels_and_Cross-referencing\#The_hyperref_package}{\lstinline!\\autoref!
del paquete \lstinline!hyperref!} con la ventaja de que nos pone la
palabra correcta en todos los casos, ya sean tablas, figuras o
ecuaciones sin necesidad de definir un comando para cada uno.

\section{Grupos de ecuaciones}\label{sec:gruposEc}

Otro tema interesante es poder escribir un grupo de ecuaciones que
comparta la misma etiqueta. Esto es posible (¡como todo en LaTeX!)
gracias a diferentes entornos aunque yo solo voy a hablar de mi
favorito: \lstinline!align! del paquete \lstinline!amsmath!. Nos permite
crear un sistema de ecuaciones que alineará según el símbolo que
marquemos con un ampersand. Por ejemplo, las 3 leyes de la termodinámica
quedarían así, alineadas según el símbolo de igual:

\begin{lstlisting}[language={[latex]tex}]
\begin{align}
  \Delta U &= Q -W \\
  \delta S &= T \mathrm{d}S \\
  S &=\mathrm{k_B}\ln\Omega
\end{align}
\end{lstlisting}

\begin{align*}
  \Delta U &= Q -W \\
  \delta S &= T \mathrm{d}S \\
  S &=\mathrm{k_B}\ln\Omega
\end{align*}

Al igual que hacíamos en el entorno \lstinline!equation!, con
\lstinline!align! también podemos añadir una etiqueta o usar el
asterisco para que no nos numere la ecuación.

\section{Formato}\label{sec:formato}

Evidentemente, LaTeX nos permite adaptar el formato de nuestras
ecuaciones a nuestros gustos o exigencias externas (véase formato de
revistas científicas, normas ISO \ldots{}). Un formato muy típico es el
siguiente:

\begin{itemize}
\item
  \emph{Cursiva para las variables}: LaTeX nos lo hace por defecto
\item
  \emph{Operadores y constantes rectos}: para los operadores del propio
  LaTeX como \lstinline!\sin! o \lstinline!\log! no tenemos que hacer
  nada, los endereza de por sí. Para el resto tenemos dos opciones: usar
  \lstinline!\mathrm! o definirlos como operadores en el preámbulo con
  \lstinline!\DeclareMathOperator! del paquete \lstinline!amsmath!. De
  esto último hablaremos más adelante, pero como sé que sois ansiosos os
  pongo cómo se haría:
\end{itemize}

\begin{lstlisting}[language={[latex]tex}]
\usepackage{amsmath}
\DeclareMathOperator{\comando}{descripción}
\end{lstlisting}

\begin{itemize}
\item
  \emph{Matrices y vectores en negrita}: para ello usaremos
  \lstinline!\mathbf! para las letras y \lstinline!\boldsymbol! para los
  símbolos o letras griegas.
\end{itemize}

Un ejemplo con todos ellos podría ser la definición de la matriz de
rigidez para el método de los elementos finitos (me sale el ingeniero
mecánico interior):

\begin{lstlisting}[language={[latex]tex}]
\begin{equation}
  \mathbf{K}=\int_V \mathbf{B}^\intercal \mathbf{D B}\mathrm{d}x\mathrm{d}y \mathrm{d}z
\end{equation}
\end{lstlisting}

que quedaría algo de este estilo:

\begin{equation*}
  \mathbf{K}=\int_V \mathbf{B}^\intercal\mathbf{D B}\mathrm{d}x \mathrm{d}y \mathrm{d}z
\end{equation*}

\section{Referencias}\label{sec:referencias}

\href{https://en.wikibooks.org/wiki/LaTeX/Mathematics}{\emph{LaTeX/Mathematics}
en WikiBooks}

\href{https://www.sharelatex.com/learn/List_of_Greek_letters_and_math_symbols}{\emph{List
of Greek letters and math symbols} en ShareLaTeX}

\href{http://latex.wikia.com/wiki/Matrix_environments}{\emph{Matrix
environments} en LaTeX Wiki}

\href{https://www.sharelatex.com/learn/Brackets_and_Parentheses}{\emph{Brackets
and Parentheses} en ShareLaTeX}

\href{https://www.sharelatex.com/learn/Operators}{\emph{Operators} en
ShareLaTeX}

\href{http://tex.stackexchange.com/questions/74353/what-commands-are-there-for-horizontal-spacing\#74354*}{\emph{What
commands are there for horizontal spacing?} en StackExchange}

\href{http://www.colorado.edu/physics/phys4610/phys4610_sp15/PHYS4610_sp15/Home_files/LaTeXSymbols.pdf}{\emph{Lista
de símbolos matemáticos} (pdf)}

\href{http://tex.stackexchange.com/questions/32100/what-does-each-ams-package-do}{\emph{What
does each AMS package do?} en StackExchange}

\href{http://moser-isi.ethz.ch/docs/typeset_equations.pdf}{\emph{How to
typeset equations in LaTeX} (pdf)}


\chapter{A vueltas con el idioma}\label{ch:idioma}
\input{Contenido/06.Idioma}

\chapter{Formas, tamaños y colores}
\input{Contenido/07.Formato}

\chapter{La página}
\input{Contenido/08.Pagina}

\chapter{Espacio en blanco}\label{ch:blanco}
\input{Contenido/09.Espacio}

\chapter{Un documento científico}
\input{Contenido/10.DocumentoCientifico}

\chapter{Píntame ese código}
\input{Contenido/11.Codigo}

%\chapter{También podemos presentar}
%Como muchos ya sabréis, con LaTeX además de fabricar documentos con una
excelente calidad también podemos crear presentaciones. Para ello
tenemos varias clases diferentes,
\href{https://www.ctan.org/pkg/beamer}{\lstinline!beamer!} es la más
famosa y probablemente habréis oído hablar de ella, pero también tienen
el mismo objetivo
\href{http://www.ctan.org/pkg/powerdot/}{\lstinline!powerdot!} y las más
viejecillas \href{http://www.ctan.org/pkg/prosper}{\lstinline!prosper!},
\href{https://www.ctan.org/pkg/seminar}{\lstinline!seminar!} y
\href{http://www.ctan.org/pkg/slides}{\lstinline!slides!}. Yo voy a
hablar de la clase \lstinline!beamer! que es la que controlo, pero antes
de nada vamos a ver en qué nos beneficia usar LaTeX para hacer una
presentación.

\section{Merece la pena usar LaTeX para una
presentación?}

He de reconocer que odio Power Point, Impress y todo el software similar
y que la primera vez que usé LaTeX para una presentación fue única y
exclusivamente por llevar la contraria, pero no volvería atrás. Estas
son las ventajas que le veo:

\begin{itemize}
\item
  \textbf{Contenido y formato separados}: esta es una de las
  características fundamentales de LaTeX y aquí nos resulta
  especialmente útil, definimos ambas cosas por separado y se afectan
  muy poco entre sí.
\item
  \textbf{Orden lógico}: nos vemos obligados a escribir el contenido
  como si fuera un texto y no como unos cuadrados con cosas dentro.
\item
  \textbf{Formato favorable para el espectador}: es más complicado poner
  muchísimo texto o imágenes sin ton si son en una diapositiva que
  hacerla sencilla y clara.
\item
  \textbf{Texto plano}: como siempre, trabajamos con texto plano por lo
  que no necesitamos un programa específico\footnote{Luego veremos que a
    la hora de presentar tal vez necesitemos un programa si queremos
    usar alguna funcionalidad específica.}, el resultado no depende del
  sistema operativo\footnote{Algo que importante cuando eres
    \emph{linuxera} en entorno Windows y no quieres que te echen la
    bronca porque \emph{sus} formatos privativos no funcionan en tu
    sistema operativo libre o viceversa, porque tus formatos estándar no
    funcionen en \emph{su} sistema privativo.}, la colaboración más
  sencilla y demás ventajas habituales del texto plano que ya conocemos.
\item
  \textbf{Reutilización}: si la presentación deriva de otro documento,
  como un artículo o tesis, que hemos escrito en LaTeX podemos copiar el
  trozo correspondiente a las imágenes, ecuaciones, tablas\ldots{}
  directamente en la presentación.
\end{itemize}

También tiene, evidentemente, sus inconvenientes:

\begin{itemize}
\item
  \textbf{No vemos lo que hacemos}: esto nos lleva pasando mucho tiempo
  pero puede ser un problema para una presentación ya que es algo más
  visual. Este problema es especialmente acuciante si no tenemos claro
  el orden en el que queremos decir las cosas.
\item
  \textbf{Diseños complejos}: es bastante difícil crear una diapositiva
  con muchos elementos y que siga teniendo buena pinta. Esto quiere
  decir que nunca conseguiríamos reproducir las míticas presentaciones
  comerciales que tienen en cada página el logo de la empresa y su
  slogan, un índice de contenidos, diecisiete imágenes, dos tablas y
  texto en tres tipos de fuente diferen te. A LaTeX le va más el
  minimalismo\footnote{Por cierto, me encanta cuando dicen que con LaTeX
    se crean presentaciones \emph{de calidad} y el ejemplo tiene unos
    colores que hacen sangrar los ojos. Aberraciones estéticas se pueden
    cometer por mucho que usemos LaTeX.}.
\end{itemize}

\section{La clase beamer}

Bien, pasemos entonces a hablar sobre \lstinline!beamer!. Aunque esta
clase tenga un
\href{http://osl.ugr.es/CTAN/macros/latex/contrib/beamer/doc/beameruserguide.pdf}{manual
de casi 250 páginas} enseguida puede uno montarse una presentación
decente. Esto se debe a que las tablas, imágenes,
bibliografía, apéndices\footnote{Para tener apéndices numerados
  necesitamos cargar el paquete \lstinline!appendixnumberbeamer!},
ecuaciones y demás características de LaTeX funcionan exactamente igual,
aunque su aspecto varía dependiendo del estilo que estemos utilizando.
Del mismo modo, \lstinline!\maketitle! nos hace la portada y
\lstinline!\tableofcontents! nos fabrica el índice de contenidos como
viene siendo habitual.

En definitiva, creamos el documento de la misma manera que creábamos un
artículo o libro, definiendo el contenido de cada diapositiva dentro del
entorno \lstinline!frame!:

\begin{lstlisting}[language={[latex]tex}]
\begin{frame}{Título}
  % Contenido de la diapositiva
\end{frame}
\end{lstlisting}

Algo a tener en cuenta es que tanto las secciones como las subsecciones
añaden una entrada al índice pero no establecen el título de la
diapositiva, debemos hacerlo nosotros a mano. Dependiendo del estilo,
tener diferentes secciones y subsecciones nos permite crear diapositivas
de título para separar cada sección y que nuestra presentación pueda
seguirse más fácilmente.

\subsection{Estilo}

En cuanto al estilo debemos diferenciar dos cosas: los \textbf{temas} y
el \textbf{estilo de ciertos elementos} como las alertas, los ejemplos o
los teoremas, que cada tema redefine.

El \textbf{tema} es lo que establece el formato general de nuestra
presentación, vendría a ser como la plantilla. Aparte de los temas que
define el propio \lstinline!beamer! y de los que hablaremos a
continuación, tenemos muchos temas disponibles en Internet, por ejemplo,
en
\href{https://www.overleaf.com/latex/templates/tagged/presentation}{Overleaf}.
A mí me gustan especialmente
\href{https://www.overleaf.com/9480607mxqxvhczzvhr\#/34336424/}{\emph{Presento}},
\href{https://www.overleaf.com/9480660qtjkqtjfqhny\#/34336601/}{el de la
universidad de Berkeley} y
\href{https://www.ctan.org/pkg/beamertheme-metropolis}{Metropolis}, el
que usé con algunas modificaciones para la defensa de mi tesis.

Lo más importante que hay que entender respecto a los temas de
\lstinline!beamer! es que hay cinco tipos:

\begin{itemize}
\item
  \textbf{Temas de presentación}: afectan a toda la presentación. Eligen
  un tema de color, uno de fuente, uno exterior y otro interior que
  combinen (relativamente) bien. Tienen nombres de ciudades.
  Antiguamente se llamaban cosas tipo \lstinline!bars! o
  \lstinline!shadow!.
\item
  \textbf{Temas de color}: afectan a la paleta de colores de la
  presentación. Hay temas de color \emph{exterior}, con nombres de
  animales acuáticos; \emph{interior}, con nombres de flores; y
  completos con nombre de animales voladores. Los temas interiores
  afectan al color de lo de dentro de la diapositiva; los exteriores a
  los elementos del borde y los completos a ambas cosas.
\item
  \textbf{Temas de fuente}: controlan el tipo de fuente que usamos en la
  presentación, por defecto es Sans Serif, pero hay opción de Serif
  (\lstinline!serif!); títulos en negrita (\lstinline!structurebold!),
  títulos en cursiva (\lstinline!structureitalicserif!) y títulos en
  versalita (\lstinline!structuresmallcapsserif!).
\item
  \textbf{Temas interiores}: controlan el aspecto de los elementos del
  interior de la diapositiva, es decir, como se muestran los bloques,
  las tablas, las figuras, las listas\ldots{} Las opciones son
  \lstinline!default!, \lstinline!circles!, \lstinline!rectangles!,
  \lstinline!rounded! e \lstinline!inmargin!. Lo más fácil es probar uno
  mismo que hace cada una, si no se hace muy largo de explicar.
\item
  \textbf{Temas exteriores}: controlan el aspecto de los bordes, es
  decir, el encabezamiento, el pie y la barra lateral. Las opciones son
  \lstinline!default!, \lstinline!miniframe!, \lstinline!sidebar!,
  \lstinline!split!, \lstinline!shadow!, \lstinline!tree! y
  \lstinline!smoothtree!. Podéis probar a ver cuál os gusta más.
\end{itemize}

Podemos combinarlos como nos parezca más bonito. Existen, de hecho,
\href{https://hartwork.org/beamer-theme-matrix/}{matrices} recopilando
combinaciones de estilos, sobre todo para los temas de presentación y de
color.

Para terminar con los temas veamos como se establecen todos ellos:

\begin{lstlisting}[language={[latex]tex}]
% Preámbulo
\usetheme{Bergen} % tema de presentación
\usecolortheme{rose} % tema de color
\usefonttheme{serif} % tema de fuente
\useinnertheme{circles} % tema interior
\useoutertheme{split} % tema exterior
\end{lstlisting}

En el caso de que haya la posibilidad de definir más opciones las
añadimos entre corchetes como argumento opcional, eso ya depende de cada
tema.

En cuanto al \textbf{estilo de los diferentes elementos} de la
presentación, son los temas los que establecen cómo son todos y cada uno
de ellos, por lo que si no nos gusta, por ejemplo, la pinta que tienen
las listas podemos pisar su estilo por defecto el nuestro. Esto nos
permite que nuestra presentación sea coherente y que no tengamos en la
diapositiva 7 el título verde y en la 23 rosa. En el siguiente apartado
veremos cómo se modifica el estilo de los elementos.

\subsection{Opciones}

Lo que sí cambia respecto a otros documentos de LaTeX es el modo de
establecer las opciones, ya que para ello usamos la familia de comandos
\lstinline!\setbeamer! y especificamos qué elemento queremos cambiar y
cómo. Según qué queramos conseguir tenemos diferentes comandos:

\begin{itemize}
\item
  \lstinline!\setbeameroption{opción general}!, establece las opciones
  generales para la presentación. Por ejemplo y tal y como veremos en la
  próxima sección, se usa para decirle a \lstinline!beamer! que muestre
  u oculte las notas mediante \lstinline!\setbeameroption{hide notes}!.
\item
  \lstinline!\setbeamertemplate{elemento}{definición}!, define el
  aspecto de cierto elemento, por ejemplo, con
  \lstinline!\setbeamertemplate{itemize   item}{$\Rightarrow$}!
  conseguimos que en las listas no numeradas se indiquen los ítems con
  una flecha.
\item
  \lstinline!\setbeamercolor{elemento}{fg=colorPrimerPlano, bg=colorDeFondo}!,
  establece el color de determinado elemento, por ejemplo,
  \lstinline!\setbeamercolor{title}{fg=magenta, bg=white}! establece que
  todos los títulos sean rosas con el fondo blanco. No es necesario usar
  las opciones \lstinline!fg! y \lstinline!bg! a la vez, lo que no
  cambiemos mantendrá el color que tenía.
\item
  \lstinline!\setbeamerfont{elemento}{size=tamaño, shape=estilo}!,
  establece la forma y tamaño de fuente de determinado elemento, por
  ejemplo, \lstinline!\setbeamerfont{title}{series=\bfseries}! pone en
  negrita el título de la presentación. Podemos establecer solo el
  tamaño o solo el estilo, lo que no cambiemos permanecerá como estaba.
\end{itemize}

Para ver qué elementos tiene una presentación no nos queda otra que
acudir al
\href{http://osl.ugr.es/CTAN/macros/latex/contrib/beamer/doc/beameruserguide.pdf}{manual},
pero ahora ya sabemos mucho y lo podemos entender perfectamente. También
en el manual encontraremos los argumentos opcionales de estos comandos.

Podemos usar todos estos comandos en cualquier parte del documento y
afectan desde donde están situados hasta encontrarse con otra definición
o, si no hay ninguna más, hasta el final.

\subsection{Notas}

Con \lstinline!beamer! tenemos la opción de crear unas notas secretas
que solo vemos nosotros en la línea de la
%\href{https://wiki.openoffice.org/wiki/Presenter_Screen}{consola del presentador} de Impress. Para escribir las notas usamos el comando
\lstinline!\note{}! que nos crea una página de notas detrás de la
diapositiva en cuestión:

\begin{lstlisting}[language={[latex]tex}]
\documentclass[notes=show]{beamer}
\begin{document}
  \begin{frame}
    % Contenido de la diapositiva
    \note{Notas}
  \end{frame}
\end{document}
\end{lstlisting}

Esto es interesante, pero se le puede sacar mucho más jugo uniéndolo al
paquete \href{http://ctan.org/pkg/pgf}{\lstinline!pgfpages!} que nos
permite unir la diapositiva con la página de notas en una hoja más ancha
de tal manera que al proyectarla nosotros veamos las notas y la
audiencia la presentación. Hay que tener en cuenta que esta
funcionalidad no es compatible con todos los visores de \emph{pdf}, en
el siguiente apartado hablaremos de ello.

Controlamos el comportamiento de las notas mediante las siguiente
opciones:

\begin{itemize}
\item
  \lstinline!\setbeameroption{hide notes}! solo muestra la presentación.
\item
  \lstinline!\setbeameroption{show only notes}! solo muestras las notas.
\item
  \lstinline!\setbeameroption{show notes on second screen=right}! crea
  una presentación el doble de ancha que contiene las diapositivas y las
  notas. En este caso mostramos las notas en la pantalla de la derecha,
  con \lstinline!left! las pondríamos en la izquierda.
\end{itemize}

Veamos como quedaría:

\begin{lstlisting}[language={[latex]tex}]
\documentclass{beamer}

\usepackage{pgfpages}
\setbeameroption{show notes on second screen=right}

\begin{document}
  \begin{frame}
    % Contenido de la diapositiva
    \note{Notas}
  \end{frame}
\end{document}
\end{lstlisting}

\subsection{Efectos y multimedia}

Que hagamos la presentación con \lstinline!beamer! no significa que vaya
a ser aburrida y estática, podemos personalizar cómo va apareciendo el
contenido e incluso añadir multimedia. Os cuento ahora unas cosillas al
respecto.

\paragraph{Overlay}

Mediante las opciones de \emph{overlay} se puede controlar cuando se
muestra cada elemento. Esto nos viene bien, por ejemplo, para mostrar
los elementos de una lista uno a uno. Para ello LaTeX nos creará
múltiples copias de la diapositiva con \emph{overlays} que mostrarán los
elementos que vayamos indicando. De esta manera, al ir avanzando dará la
sensación de que va \emph{surgiendo} o \emph{desapareciendo} contenido
en la presentación.

Hay varios comandos para gestionar este mecanismo, como los que siguen:

\begin{itemize}
\item
  \lstinline!\pause!: solo se muestra el contenido hasta este punto.
\item
  \lstinline!\uncover<OVERLAY>{CONTENIDO}!: solo se muestra el contenido
  en las copias indicadas pero se le reserva espacio desde el principio.
\item
  \lstinline!\only<OVERLAY>{CONTENIDO}!: como \lstinline!\uncover! pero
  sin que se reserve espacio previamente para el contenido.
\end{itemize}

Además, muchos otros comandos y entornos aceptan opciones de
\emph{overlay}, generalmente con esta estructura:

\begin{lstlisting}[language={[latex]tex}]
\comando<OVERLAY>{CONTENIDO}

\begin{entorno}<OVERLAY>
CONTENIDO
\end{entorno}
\end{lstlisting}

En \lstinline!OVERLAY! especificamos cuándo queremos que aparezcan las
cosas, \lstinline!<1>! mostrará el contenido solo en la primera copia;
\lstinline!<2->! en todas a partir de la segunda y \lstinline!<-5>! solo
hasta la quinta.

Un caso interesante es el de las listas, ya que hay una sintaxis
simplificada para mostrar los elementos de uno en uno:

\begin{lstlisting}[language={[latex]tex}]
\begin{itemize}[<+->]
 \item Ítem 1
 \item Ítem 2
\end{itemize}
\end{lstlisting}

Aprovecho el ejemplo para decir que los \emph{overlays} también
funcionan en las notas:

\begin{lstlisting}[language={[latex]tex}]
\note<1>{Notas para el ítem 1}
\note<2>{Notas para el ítem 2}
\end{lstlisting}

\paragraph{Vídeos}
En las presentaciones de \lstinline!beamer! también podemos añadir
vídeos, faltaría más. Hay varios paquetes con este fin, como
\lstinline!multimedia!, que viene con el propio \lstinline!beamer! y
\href{https://www.ctan.org/pkg/media9}{\lstinline!media9!}, que
sustituye a
\href{https://www.ctan.org/pkg/movie15}{\lstinline!media15!}. El
problema aquí es que muy pocos visores de \emph{pdf} soportan los vídeos
incrustados. Si conseguís encontrar uno, los vídeos se incrustan muy
fácilmente:

\begin{lstlisting}
% Vídeo con multimedia
\movie[OPCIONES]{SUSTITUTO}{VÍDEO}

% Vídeo con media9
\includemedia[OPCIONES]{SUSTITUTO}{VÍDEO}
\end{lstlisting}

Donde \lstinline!SUSTITUTO! es el texto o imagen que guardará sitio al
vídeo, por ejemplo, un fotograma del mismo. En \lstinline!VÍDEO! debemos
escribir la ruta al vídeo.

Otra opción es usar el comando \lstinline!\href! del paquete
\href{https://www.ctan.org/pkg/hyperref?lang=en}{\lstinline!hyperref!},
que sirve para crear enlaces en nuestros documentos. En lugar de
incrustar el vídeo en la presentación, en este caso ponemos un texto o
imagen para que cuando la pinchemos se abra el reproductor de vídeo en
una ventana aparte:

\begin{lstlisting}
\href{run:VIDEO}{SUSTITUTO}
\end{lstlisting}

\paragraph{Navegación}

En la parte inferior de las diapositivas nos aparecen por defecto unos
iconitos para navegar por la presentación y buscar. Hay gente que los
ama y gente que los detesta. Si sois de los segundos podéis asesinarlos
con:

\begin{lstlisting}[language={[latex]tex}]
\setbeamertemplate{navigation symbols}{}
\end{lstlisting}

\paragraph{Repetición}

La última cosa que os voy a contar sobre \lstinline!beamer! es cómo
insertar automáticamente un contenido concreto cuando se dé cierto
evento. Esto es útil para hacer aparecer una diapositiva con el título o
para mostrar el índice cada vez que vaya a comenzar una nueva sección,
por poner un par de ejemplo.

Conseguimos esto con la familia de comandos\lstinline!\AtBegin! (y
\lstinline!\AtEnd! para el caso de las notas) que se disparan al
comenzar una sección, parte, nota o demás. Os dejo aquí dos casos que
creo que se entienden con facilidad:

\begin{lstlisting}[language={[latex]tex}]
% Diapositiva con el título de sección al iniciar sección
\AtBeginSection{
  \begin{frame}
  \vfill
  % Caja con colores de título
  \begin{beamercolorbox}[center]{title}
    \usebeamerfont{title} % Fuente de título
    \insertsectionhead % Nombre de sección
  \end{beamercolorbox}
  \vfill
  \end{frame}
}

% Índice mostrando subsección actual al iniciar subsección
\AtBeginSubsection
{   \begin{frame}{Outline}
        \tableofcontents[currentsection,
        currentsubsection,
        sectionstyle=show/hide,
        subsectionstyle=show/shaded/hide] 
    \end{frame}
}
\end{lstlisting}

\section{Programas para presentar}

El mayor problema de \lstinline!beamer! desde mi punto de vista es que
no todos los visores de \emph{pdf} son capaces de mostrarnos en nuestro
ordenador las notas y proyectar las diapositivas. Si somos valientes y
damos las presentaciones a pelo esto no nos importa, con cualquier
lector en pantalla completa estamos servidos, pero si somos cobardicas
con miedo escénico como la que escribe tenemos un problema.

No nos asustemos aún! El mundo es grande y los cobardes que usan LaTeX
y saben programar parece que abundan. Es por ello que hay diferentes
alternativas para que podamos hacer trampa y leer de nuestras notas
secretas. En concreto voy a hablar de \emph{pdfpc} que es el que yo he
usado, luego nombraré algunos otros que sé que existen pero poco más.

\subsection{Pdfpc}

\href{https://pdfpc.github.io/}{\emph{Pdfpc}} es una herramienta de
línea de comandos para visualizar presentaciones en formato \emph{pdf}
en varias pantallas. Es un \emph{fork} de \emph{Pdf Presenter Console},
que \href{https://github.com/jakobwesthoff/Pdf-Presenter-Console}{dejó
de desarrollarse}. Se distribuye con licencia
\href{https://github.com/pdfpc/pdfpc/blob/master/LICENSE.txt}{GNU GPL
v2} así que es software libre. Es muy fácil de utilizar y ayuda mucho a
la hora de presentar, no solo por las notas, como luego veremos.

La única pega que le pondría es que tuve que compilarlo desde fuente
porque el que estaba en los repositorios era muy viejecito, pero no es
difícil, yo lo hice en GNU/Linux y hasta en Windows con Cygwin\footnote{Hablé
  un poco más en detalle sobre cómo compilar
  \href{https://ondahostil.wordpress.com/2016/10/24/lo-que-he-aprendido-compilar-pdf-presenter-console-con-cygwin/}{aquí}}.

Para usarlo simplemente escribimos:

\begin{lstlisting}[language={[latex]tex}]
pdfpc PRESENTACIÓN
\end{lstlisting}

Donde \lstinline!PRESENTACIÓN! es la ruta a la presentación en
\emph{pdf}.

De por sí \emph{pdfpc} nos enseña en la vista de presentador la
diapositiva actual, la siguiente, un reloj, el número de la diapositiva
actual y el total. Todo ello muy útil a la hora de presentar.

\begin{figure}[htbp]
\centering
\includegraphics[width=\textwidth]{docs/Figuras/pdfpc.png}
\caption{pdfpc en acción}
\end{figure}

Para ver las notas de \lstinline!beamer! necesitamos crear la
presentación con las notas integradas como hemos visto antes:

\begin{lstlisting}
\setbeameroption{show notes on second screen=right}
\end{lstlisting}

Luego llamamos a \emph{pdfpc} con la opción \lstinline!--notes!:

\begin{lstlisting}[language=bash]
pdfpc presentation.pdf --notes=right
\end{lstlisting}

Ahora en la vista de presentador veremos las notas y una minidiapositiva
mostrándonos la diapositiva actual, en lugar de verla en grande como
antes.

\begin{figure}[htbp]
\centering
\includegraphics[width=\textwidth]{docs/Figuras/pdfpcNotas.png}
\caption{pdfpc con notas}
\end{figure}

El programa tiene otras muchas opciones, os resumo unas pocas que me
parecen especialmente útiles, las demás están en el manual:

\begin{itemize}
\item
  \lstinline!-d!, \lstinline!--duration=N! la duración en minutos
  (\lstinline!N!) de la presentación. Sirve para que nos ponga una
  cuenta atrás en la parte inferior de la pantalla.
\item
  \lstinline$-l, \lstinline!--last−minutes=N$ tiempo en minutos
  (\lstinline!N!) a partir del que la cuenta atrás se verá en rojo. Para
  irse poniendo nerviosillo.
\item
  \lstinline!-s!, \lstinline$--switch−screens$ cambia la vista de
  presentador de pantalla.
\item
  \lstinline!-w!, \lstinline!--windowed! crea dos ventanas, una con la
  vista del presentador y otra con lo que verá la audiencia. Útil para
  ver el resultado cuando solo tenemos una pantalla.
\end{itemize}

Por ejemplo, para presentación de la tesis usé lo siguiente\footnote{Y
  no llegué a la cuenta atrás en rojo porque en media hora lo tenía
  ventilado.}:

\begin{lstlisting}[language=bash]
pdfpc presentation.pdf --duration=45 --notes=right --last-minutes=10
\end{lstlisting}

Además, durante la presentación se pueden usar diferentes teclas para
\emph{hacer cosas}:

\begin{itemize}
\item
  \lstinline!F! (\emph{freeze}): congela la imagen de la presentación
  para la audiencia mientras nosotros jugamos en nuestra vista. Pinta un
  copo de nieve en la parte inferior.
\item
  \lstinline!B! (\emph{black}): pone la pantalla de la audiencia negra y
  a nosotros nos pinta un cuadradito negro con una cruz blanca. Útil
  cuando das clase y alternas pizarra y proyector (así no montas el lío
  que solían montar mis profesores, ingenieros industriales casi todos
  ellos).
\item
  \lstinline!G! (\emph{go}): nos lleva a la diapositiva que le
  indiquemos. Fantástico para cuando te dicen \emph{en la diapositiva 12
  hay una tabla que\ldots{}}
\item
  \lstinline!N! (\emph{notes}): nos permite escribir notas en la
  diapositiva. Salimos con \lstinline!ESC!.
\item
  \lstinline!E! (\emph{end}): marca la diapositiva final. Útil si
  tenemos \emph{diapositivas de repuesto} para las preguntas.
\item
  \lstinline!O! (\emph{overlay}): sirve para marcar/desmarcar
  diapositivas como parte de una diapositiva que va surgiendo poco a
  poco. No las tendrá en cuenta en el cómputo total de diapositivas.
\item
  \lstinline!P! (\emph{pause}): pausa el reloj.
\item
  \lstinline!R! (\emph{reset}): reinicia la presentación.
\item
  \lstinline!Q! (\emph{quit}) o \lstinline!ESC!: cierra la presentación.
\end{itemize}

Las notas y diferentes marcas (fin, \emph{overlay}, \ldots{}) las guarda
en un archivo \emph{pdfpc} que recupera cada vez que leemos la
presentación. Es un archivo de texto plano y podemos abrirlo. Tiene esta
pinta:

\begin{lstlisting}
[file]
presentation.pdf
[duration]
45
[skip]
8,
[end_ser_slide]
10
[notes]
### 1
Notas en la diapositiva 1
\end{lstlisting}

\subsection{Otras opciones para
presentar}

Los programas que cito ahora nunca los he usado, encontré \emph{pdfpc} y
me quedé con él, los pongo aquí para que vosotros elijáis el que más os
guste.

\begin{itemize}
\item
  \href{http://impressive.sourceforge.net/}{\emph{Impressive}}\footnote{Gracias
    a \href{https://quitter.se/notice/9129937}{Shevek} por informarme de
    su existencia!}: un programa escrito en Python con funcionalidades
  curiosas como el modo foco y la vista global de todas las
  diapositivas.
\item
  \href{https://github.com/Cimbali/pympress}{\emph{Pympress}}: también
  escrito en Python, soporta vídeo y las notas de \lstinline!beamer! y
  tiene una consola para el presentador.
\item
  \href{https://github.com/dannyedel/dspdfviewer}{\emph{Dspdfviewer}}:
  un visor simple para las presentaciones de \lstinline!beamer!.
\end{itemize}

\section{Resumen}

Para resumir lo que hemos estado comentando, vamos a ver cómo quedaría
un ejemplo completo aunque sencillo de una presentación en LaTeX:

\begin{lstlisting}[language={[latex]tex}]
% Definición
\documentclass{beamer}

% Notas
\usepackage{pgfpages}
\setbeameroption{show notes on second screen=right}

% Datos
\title{Presentaciones en \LaTeX}
\author{Ondiz}
\institute{Home, sweet home}
\date{\today}

% Temas
\usetheme{Bergen}
\usefonttheme{serif}
\usecolortheme{rose}

% Opciones
\setbeamercolor{title}{fg=magenta, bg=white}
\setbeamertemplate{navigation symbols}{}

% Inicio
\begin{document}

% Diapositivas
 \begin{frame}
  \maketitle
  \note{Notas}
 \end{frame}
 
 \begin{frame}{Índice}
  \tableofcontents
  \note{Más notas}
 \end{frame}
 
 \section{Introducción}
 \subsection{Primera parte}

 \begin{frame}{Introducción}
  \begin{itemize}
   \item<1-> Ítem 1
   \item<2> Ítem 2
  \end{itemize}
 \end{frame}

\end{document}
\end{lstlisting}

No es tan difícil!

En cualquier caso, para empezar con \lstinline!beamer! yo recomendaría
coger una presentación ya hecha y probar a cambiar cosas hasta que nos
sintamos cómodos con este nuevo sistema de trabajo. Cuando ya manejemos
lo más sencillo ir al manual (o a StackOverflow) y personalizar la
presentación es coser y cantar.

Por último, como soy maja os dejo una presentación de
\href{https://github.com/Ondiz/cursoLatex/tree/master/Ejemplos/Presentacion}{ejemplo}
en el repositorio del curso, contiene muchas de las cosas que he
comentado.

\section{Referencias}

\href{https://tex.stackexchange.com/questions/16204/which-package-to-use-for-presentations-beamer-prosper-or-other}{\emph{Which
package to use for presentations? Beamer, Prosper, or Other} en
TeXExchange}

\href{http://www.dmi.me.uk/blog/2010/11/08/creating-a-presentation-with-latex-and-powerdot/}{\emph{Creating
a presentation with LaTeX and powerdot}}

\href{http://osl.ugr.es/CTAN/macros/latex/contrib/beamer/doc/beameruserguide.pdf}{Manual
de \lstinline!beamer!}

\href{https://tug.org/pracjourn/2005-2/miller/miller.pdf}{\emph{Producing
beautiful slides with LaTeX}}

\href{https://hartwork.org/beamer-theme-matrix/}{\emph{Beamer theme matrix}}

%\href{http://www.deic.uab.es/~iblanes/beamer_gallery/index.html}{\emph{Beamer theme gallery}}

\href{https://www.r-bloggers.com/create-your-own-beamer-template/}{\emph{Create
your own beamer template}}

\href{https://tex.stackexchange.com/questions/1574/embedding-videos-and-animations}{\emph{Embedding
videos and animations} en TeXExchange}

\href{https://tex.stackexchange.com/questions/21777/is-there-a-nice-solution-to-get-a-presenter-mode-for-latex-presentations}{\emph{Is
there a nice solution to get a ``presenter mode'' for Latex
presentations?} en TeXExchange}


\chapter{Nuestras propias macros}
Hoy vamos a salir de mi zona de confort y hablar sobre la creación de
\href{http://foldoc.org/macro}{\emph{macros}}, es decir, de nuevos
comandos y entornos\footnote{¡Líos de nomenclatura a la vista! Dijimos
  hace mucho que LaTeX es un \emph{conjunto de macros} para TeX. Luego
  hemos separado estas \emph{macros} en comandos y entornos, pero un
  entorno no deja de ser un conjunto de comandos que afecta de forma
  local. Llamaremos también \emph{macro} a los comandos (y por extensión
  entornos) que \emph{definamos nosotros}, tal y como se suele hacer en
  el mundillo.}. No soy ninguna experta en esto, pero hay un par de
ideas que me parece que hay que tener claras a la hora de definir cosas
en LaTeX. Básicamente voy a contar lo que me hubiera gustado que me
contaran cuando empecé con esto, más que nada para no copiar de
StackOverflow a ciegas.

Lo primero y más importante que tenemos que saber a la hora de jugar con
las macros en LaTeX es que tenemos dos opciones
\footnote{Para ser sinceros también está \lstinline!\\providecommand!, que crea el comando si no existe y si no ignora la definición, pero no tiene un\href{https://tex.stackexchange.com/questions/56667/why-is-there-no-provideenvironment}{primo para los entornos}.}:

\begin{itemize}
\item
  \textbf{Crear un entorno o comando desde cero}. Así conseguimos que
  LaTeX haga algo que no hacía o guardamos un conjunto de órdenes que
  usamos a menudo en una macro con el objetivo escribir menos. La
  palabra clave para esto es \emph{new}.
\item
  \textbf{Pisar un entorno o comando existente}. En este caso la idea es
  modificar el comportamiento de cierto comando o entorno a nuestro
  gusto. Se conoce como \emph{renew}.
\end{itemize}

El siguiente concepto en orden de importancia es que podemos (re)definir
comandos en cualquier parte del documento, pero para tener todo
perfectamente organizado es preferible hacerlo en el preámbulo.

Veamos entonces como crear comandos y entornos nuevos y modificar los
existentes. Voy a intentar que todos los ejemplos resuelvan problemas
reales, que no sean \emph{de juguete}.

\section{Escribir comandos}

Han ido apareciendo comandos nuevos\footnote{Cuando aprendimos a
  escribir
  \href{https://ondiz.github.io/cursoLatex/Contenido/05.Ecuaciones.html}{ecuaciones}
  vimos un truco para no tener que escribir la palabra \emph{Ecuación} a
  la hora de referenciar.} y trucados\footnote{Cuando hablamos del
  \href{https://ondiz.github.io/cursoLatex/Contenido/06.Idioma.html}{idioma}
  vimos cómo modificar el nombre de las tablas.} anteriormente, ¡hoy
llega por fin la explicación que os debía! Primero vamos a fabricar
comandos nuevecitos, luego modificaremos alguno que ya existe para que
sea más divertido.

\subsection{Comandos nuevos}

Definir comandos nuevos en LaTeX es sencillo, solo debemos seguir la
siguiente estructura:

\begin{lstlisting}[language={[latex]tex}]
\newcommand{COMANDO}[ARGUMENTOS]{DEFINICIÓN}
\end{lstlisting}

donde:

\begin{itemize}
\item
  \lstinline!COMANDO! será el nombre del comando que queramos definir.
  Empezará por \lstinline!\!. Solo podemos usar letras para bautizarlo.
\item
  \lstinline!ARGUMENTOS! será el número de argumentos entre 0 y 9 que le
  pasaremos al comando. Como veis, que un comando tenga argumentos es
  opcional.
\item
  \lstinline!DEFINICIÓN! será donde escribiremos lo que hace el comando.
  Haremos referencia a los diferentes argumentos mediante \lstinline!#!
  seguida del número correspondiente.
\end{itemize}

Veamos un ejemplo. Vamos a crear un comando que nos escriba
\lstinline!Figura X! en lugar de \lstinline!X! cuando hagamos referencia
a cierta figura:

\begin{lstlisting}[language={[latex]tex}]
\newcommand{\figref}[1]{\figurename~\ref{#1}}
\end{lstlisting}

Analicémoslo:

\begin{itemize}
\item
  El nombre del nuevo comando es \lstinline!\figref{}! y tiene un único
  argumento, la etiqueta de la figura, a la que hacemos referencia
  gracias a nuestro viejo conocido \lstinline!\ref{}!.
\item
  \lstinline!\figurename! es el comando que guarda el nombre de las
  figuras\footnote{\href{http://www.tex.ac.uk/FAQ-fixnam.html}{Del mismo
    modo}, el nombre de los capítulos se guarda en
    \lstinline!\\chaptername! y el de las tablas en
    \lstinline!\\tablename!. Cuidado porque no todos los elementos siguen
    este patrón, de hecho, \lstinline!\\sectionname! no existe.}.
  Podríamos escribir \emph{Figura} a mano, pero si cambiamos el idioma
  tendríamos que cambiar también la definición. De este modo LaTeX,
  sustituye \lstinline!\figurename! por el nombre de la figura según le
  mande el paquete de idioma.
\item
  Usamos un \emph{espacio duro} entre el nombre y el número para que
  LaTeX no meta en medio un salto de línea o de página.
\end{itemize}

Nuestra nueva macro se usa exactamente igual que cualquier otro comando:

\begin{lstlisting}[language={[latex]tex}]
\begin{figure}[H]
  \includegraphics[width=0.7\textwidth]{Figuras/esquema.eps}
  \caption{Esquema del proceso}
  \label{fig:esquema}
\end{figure}

Como vemos en la \figref{fig:esquema}...

% Equivalente a 
Como vemos en la Figura~\ref{fig:esquema}...
\end{lstlisting}

Un tema interesante a la hora de definir comandos es la inclusión de
\textbf{argumentos por defecto} en la definición del mismo:

\begin{lstlisting}[language={[latex]tex}]
\newcommand{COMANDO}[ARGUMENTOS][DEFECTO]{DEFINICIÓN}
\end{lstlisting}

donde \lstinline!DEFECTO! es el valor que tomará el argumento opcional
si no se especifica. El argumento opcional siempre es el primero.

Un ejemplo de uso podría ser un texto matemático en el que hagamos
referencia al plano real a menudo pero tal vez nos haga falta alguna vez
hablar de un espacio de dimensión mayor. Para ello podemos definir un
comando que nos escriba la
\href{https://en.wikipedia.org/wiki/Real_number\#/media/File:Latex_real_numbers.svg}{R
molona esa} y que por defecto el espacio sea bidimensional, pero podamos
cambiarlo opcionalmente:

\begin{lstlisting}[language={[latex]tex}]
\usepackage{amssymb}
\newcommand{\R}[1][2]{\mathbb{R}^{#1}}
\end{lstlisting}

A la hora de usarlo le pasamos el argumento opcional cuando lo
necesitamos:

\begin{lstlisting}[language={[latex]tex}]
% Espacio bidimensional
$\R$

% Espacio tridimensional
$\R[3]$
\end{lstlisting}

\subsection{Comandos trucados}

Hemos dicho al principio que además de crear nuestros propios comandos
podemos modificar el comportamiento de alguno existente. Para ello
usamos \lstinline!\renewcommand! en lugar de \lstinline!\newcommand! con
la misma sintaxis:

\begin{lstlisting}[language={[latex]tex}]
\renewcommand{COMANDO}[ARGUMENTOS]{DEFINICIÓN}
\end{lstlisting}

donde:

\begin{itemize}
\item
  \lstinline!COMANDO! será el nombre del comando que queramos modificar.
\item
  \lstinline!ARGUMENTOS! será el número de argumentos, igual que antes.
\item
  \lstinline!DEFINICIÓN! será la nueva definición del comando.
\end{itemize}

Como ejemplo de esta sección, vamos a usar \lstinline!\renewcommand!
para evitar tener que activar el modo matemático cuando escribamos
fracciones. Con este fin vamos a echar mano de
\lstinline!\ensuremath{}!,
que nos permite usar comandos matemáticos dentro y fuera de las
ecuaciones.

Como vamos a crear la nueva definición a partir del propio comando,
necesitamos guardarlo en otro sitio primero para que la definición no
sea recursiva. En este menester nos ayuda el comando
\href{https://en.wikibooks.org/wiki/TeX/let}{\lstinline!\\let!}, que
sirve para copiar el contenido de un comando en uno nuevo:

\begin{lstlisting}[language={[latex]tex}]
\let\comandoNuevo=\comandoViejo
\end{lstlisting}

Juntando las piezas, tenemos lo siguiente:

\begin{lstlisting}[language={[latex]tex}]
% Guardamos la definición original
\let\oldfrac=\frac
% Modificamos \frac para que funcione fuera de ecuaciones
\renewcommand{\frac}[2]{\ensuremath{\oldfrac{#1}{#2}}}
\end{lstlisting}

Ahora podemos usar \lstinline!\frac! directamente en el texto.

\section{Escribir entornos}

¡Ya sabemos crear y cambiar comandos! Vamos a dar un paso más y hacer
los mismo para los entornos.

\subsection{Entornos nuevos}

La sintaxis para la definición de entornos nuevos es muy similar a la de
los comandos, usando ahora \lstinline!\newenvironment!:

\begin{lstlisting}[language={[latex]tex}]
\newenvironment{ENTORNO}[ARGUMENTOS]{ANTES}{DESPUÉS}
\end{lstlisting}

En \lstinline!ANTES! escribiremos el grupo de comandos que hay que
ejecutar al iniciar el entorno y, por tanto, los que le darán el formato
al mismo, y \lstinline!DESPUÉS!, los que se activarán tras el texto. El
resto de elementos funciona como antes.

De esta manera, podemos definir un entorno para poner notas en el texto.
Yo he creado uno que rodea el texto de la nota con dos rayas
(\lstinline!\hrule!), una por debajo y una por encima, y que nos permite
darle un título, que aparecerá en negrita:

\begin{lstlisting}[language={[latex]tex}]
\newenvironment{nota}[1]
  {\vspace{1ex}\hrule\textbf{#1}}
  {\vspace{1ex}\hrule}
\end{lstlisting}

Este entorno nuevecito y reluciente se usa en el cuerpo del documento
como cualquier otro:

\begin{lstlisting}[language={[latex]tex}]
\begin{nota}{Cuidado!}
  Hay que tener en cuenta que
\end{nota}
\end{lstlisting}

Un tema interesante son los
\href{https://www.sharelatex.com/learn/Counters}{contadores} gracias a
los cuales podremos fabricar \textbf{entornos numerados}. Los contadores
tiene la estructura \lstinline!\theELEMENTO!. Así, \lstinline!\thepage!
contiene el número de página, \lstinline!\thechapter!\footnote{Uniéndolo
  con lo que hemos dicho anteriormente, \lstinline!\\thechapter! contiene
  el número del capítulo, \lstinline!\\chaptername! su nombre y
  \lstinline!\\chaptermark! el título.} el número de capítulo y
\lstinline!\theNOMBRE! el número del elemento numerado que hayamos
creado.

Para crear un contador usamos el comando \lstinline!\newcounter!, le
damos un nombre y, opcionalmente, le decimos dónde debe reiniciar la
cuenta:

\begin{lstlisting}[language={[latex]tex}]
\newcounter{NOMBRE}[REINICIO]
\end{lstlisting}

El tema del reinicio es interesante para los documentos largos, así
podemos empezar a contar al iniciar un capítulo y hacer referencia al
elemento mediante \lstinline!\thechapter.\theNOMBRE! que nos escribirá
el número del capítulo seguido del número del elemento. Con el ejemplo
que viene a continuación se entenderá mejor, espero.

Luego, incrementamos el valor del contador cuando sea necesario con:

\begin{itemize}
\item
  \lstinline!\stepcounter{CONTADOR}!: incrementa en uno el valor de
  \lstinline!CONTADOR!.
\item
  \lstinline!\refstepcounter{CONTADOR}!: incrementa en uno el valor de
  \lstinline!CONTADOR! y nos permite usarlo en las referencias cruzadas.
\item
  \lstinline!\addtocounter{CONTADOR}{NÚMERO}!: incrementa
  \lstinline!CONTADOR! en un valor que le pasemos.
\end{itemize}

Una idea que se me ocurre para hacer uso de esta funcionalidad es crear
un entorno numerado para poner ejemplos en el texto cuya numeración se
reinicie al cambiar de sección:

\begin{lstlisting}[language={[latex]tex}]
% Creamos un nuevo contador que se reinicie al cambiar de sección
\newcounter{ejemplo}[section]
% Incrementamos en uno el contador al iniciar el nuevo entorno
% Accedemos a su contenido con \theejemplo
\newenvironment{ejemplo}
  {\refstepcounter{ejemplo}\vspace{1ex}\hrule\textbf{Ejemplo~\thesection.\theejemplo}}
  {\vspace{1ex}\hrule}
\end{lstlisting}

Así, si escribimos algo de este estilo:

\begin{lstlisting}[language={[latex]tex}]
\section{Entorno numerado}
  \begin{ejemplo}
    Un primer ejemplo
  \end{ejemplo}

  \begin{ejemplo}
    Un segundo ejemplo
  \end{ejemplo}
\end{lstlisting}

Conseguiremos lo siguiente:

\begin{figure}[htbp]
\centering
\includegraphics[width=\textwidth]{docs/Figuras/entornoNum.png}
\caption{Entornos numerados}
\end{figure}

\subsection{Entornos trucados}

Al igual que modificábamos comandos existentes con
\lstinline!\renewcommand!, podemos cambiar entornos con
\lstinline!\renewenvironment!:

\begin{lstlisting}[language={[latex]tex}]
\renewenvironment{ENTORNO}[ARGUMENTOS]{ANTES}{DESPUÉS}
\end{lstlisting}

Lo que tenemos que tener en cuenta aquí es la implementación original
del entorno que vamos a cambiar. A mí me ayuda pensar en un entorno como
dos comandos, uno que da inicio al formato concreto y otro que lo
finaliza. Es decir, esto:

\begin{lstlisting}[language={[latex]tex}]
\begin{equation}
  a^2 x + b x + c = 0
\end{equation}
\end{lstlisting}

es equivalente a:

\begin{lstlisting}[language={[latex]tex}]
\equation
  a^2 x + b x + c = 0
\endequation
\end{lstlisting}

De esta manera me resulta más sencillo saber qué hay que escribir en los
argumentos \lstinline!ANTES! y \lstinline!DESPUÉS! de los que hablaba.

Para terminar con los entornos os dejo con un ejemplo complejo que monté
juntando piezas de aquí y de allí para que las citas aparecieran en gris
oscuro con una barrita gris clara a la izquierda. Es complicadillo y yo
misma no sé si lo entiendo muy bien, pero es para que veáis un caso
real:

\begin{lstlisting}[language={[latex]tex}]
\usepackage{framed}

% Redefinir leftbar
\renewenvironment{leftbar}[1][\hsize]
  {\color{gray}
    \def\FrameCommand
    {{\color{lightgray}\vrule width 3pt}}
    \MakeFramed{\hsize#1\advance\hsize-\width\FrameRestore}
  }
  {\endMakeFramed}
  
% Guardar entorno quote, lo forman dos comandos
\let\oldquote=\quote
\let\oldendquote=\endquote

% Barra vertical a la izquierda de la cita
\renewenvironment{quote}
  {\vspace{10pt}\leftbar\vspace*{-6pt}\oldquote}
  {\oldendquote\endleftbar\vspace{10pt}}
\end{lstlisting}

En fin, creo que la única manera de aprender a modificar entornos es
modificar entornos así que no queda más remedio que practicar.

\section{Una nota sobre TeX}

En el principio de los tiempos dijimos que LaTeX es un conjunto de
macros escritos en TeX (o Plain TeX) ¿recordáis? Esto provoca que LaTeX
tenga algunas limitaciones a la hora de definir cosas, limitaciones que
TeX, al ser de más bajo nivel, no tiene.

El comando \lstinline!\let! que hemos usado para guardar la definición
original de un comando que íbamos a modificar pertenece a TeX. También
\lstinline!\def! que
aparece en ejemplo de las citas personalizadas es parte de TeX y sirve
para definir comandos nuevos, al igual que
\lstinline!\newcommand!\footnote{Se pueden declarar comandos nuevos
  dentro de entornos de manera similar a declararlos de manera
  independiente.
  \href{https://en.wikibooks.org/wiki/LaTeX/Macros\#Declare_commands_within_new_environment}{Aquí}
  tenéis más información.}. Os hablo de ellos porque los veréis a menudo
cuando busquéis ejemplos por ahí.

\section{En definitiva, ¿qué hay que
saber?}

Diría que lo más importante es saber la diferencia entre \emph{definir}
(\lstinline!\new!) y \emph{pisar} (\lstinline!\renew!) un comando o
entorno existente y cuándo hay que hacer lo uno o lo otro. Tampoco está
de más recordar que escribimos tanto las definiciones como las
modificaciones en el preámbulo. Igualmente, nos viene bien saber de la
existencia de comandos de TeX como \lstinline!\let! y \lstinline!\def!
que nos hacen la vida más fácil.

Para acabar, os dejo mi proceso para crear macros:

\begin{enumerate}
\def\labelenumi{\arabic{enumi}.}
\item
  Escribo mi combinación de comandos en el cuerpo del documento y
  verifico que funciona.
\item
  Escribo cómo quiero que sea mi comando o entorno final y lo comparo
  con mi combinación de comandos.
\item
  Traslado la combinación al preámbulo y le doy forma según la sintaxis
  correspondiente.
\item
  Extraigo los argumentos y pienso si puedo darle un valor por defecto a
  alguno de ellos.
\item
  Pruebo si funciona.
\end{enumerate}

Lo dicho, a escribir macros se aprende escribiendo macros. ¡Dadle duro!

\section{Referencias}\label{referencias}

\href{http://alvinalexander.com/blog/post/latex/create-your-own-commands-in-latex-using-newcommand}{\emph{LaTeX
example: How to create your own commands with \lstinline!newcommand!}}

\href{https://en.wikibooks.org/wiki/LaTeX/Macros}{\emph{LaTeX/Macros} en
Wikibooks}

\href{http://www.shawnlankton.com/2008/01/newcommand-with-argument-in-latex/}{\emph{\\newcommand
with arguments in LaTeX}}

\href{https://tex.stackexchange.com/questions/35564/plain-tex-vs-latex-macros}{\emph{Plain
TeX vs.~LaTeX Macros} en TeXExchange}

\href{https://en.wikibooks.org/wiki/LaTeX/Plain_TeX}{\emph{LaTeX/Plain
TeX} en Wikibooks}

\href{https://tex.stackexchange.com/questions/26742/list-of-higher-level-latex-commands-corresponding-to-tex-commands/26922\#26922}{\emph{List
of higher-level LaTeX commands corresponding to TeX commands} en
TeXExchange}

\href{https://tex.stackexchange.com/questions/655/what-is-the-difference-between-def-and-newcommand}{\emph{What
is the difference between \\def and \\newcommand?} en TeXExchange}


\chapter{Abramos la caja de herramientas}
\input{Contenido/14.Herramientas}

\chapter{La opción Pandoc}
\input{Contenido/15.Pandoc}

\chapter{Mi esquema de trabajo}
Hemos aprendido a crear documentos y presentaciones en LaTeX, sabemos
escribir macros propias y convertir archivos de un tipo a otro. Os
podría contar cómo cambiar la fuente del texto o hablar de diferentes
paquetes para modificar la apariencia de nuestro documento, pero creo
con lo que ya sabéis sois muy capaces de entender por vosotros mismos
cualquier paquete leyendo el manual. Así que en este último capítulo voy
a hablar de algo que desde mi punto de vista no se trata lo suficiente:
\textbf{cómo trabajo con LaTeX} en, por llamarlo de algún modo, un
entorno de producción. Veremos cómo organizo los archivos, qué software
y paquetes utilizo y algunas cosillas sobre compilación y colaboración.

\section{¿Cómo me organizo?}

Como he dicho montones de veces \emph{la organización es fundamental},
aunque no igual para todo el mundo. Cuando el documento es corto solo
separo las imágenes en una carpeta propia y escribo el contenido en el
mismo archivo en el que defino el estilo. Si, por el contrario, se trata
de un documento más extenso, como un libro, creo un archivo de LaTeX
principal desde el que llamo a las diferentes secciones o capítulos con
\lstinline!\include{}! o \lstinline!\input{}!.

A la hora de separar los archivos, tiendo a separarlos por tipo, lo que
me facilita aplicarles a todos ellos una misma acción\footnote{Pensad en
  cambiar el formato de todas las imágenes o buscar una palabra en el
  contenido.}. En ocasiones hago una segunda clasificación por capítulos
si tengo muchas figuras o extractos de código, por ejemplo.

\begin{lstlisting}
-- principal.tex
-- estilo.bst
-- referencias.bib

-- Contenido
.. -- 1.Intro.tex
.. -- 2.Segundo.tex
.. -- ...

-- Código
.. -- listing.py
.. -- ...

-- Figuras
    -- fig.eps
    -- ...
\end{lstlisting}

Si la definición del estilo es muy larga o estoy usando una clase que he
descargado por ahí, añado una carpeta extra para meter todo eso y no
volverme loca.

En cuanto al \textbf{software}, suelo escribir en Kile con el corrector
ortográfico activado, el ancho de línea fijado a 80 caracteres\footnote{Reducir
  el ancho de línea resulta útil a la hora de ver los cambios que hemos
  llevado a cabo en determinado archivo, \lstinline!git! nos pinta la
  línea modificada entera, si es gigante será más difícil encontrar la
  palabra exacta que hemos cambiado.} y una orden personalizada para
compilar que genera el documento con referencias cruzadas y
bibliográficas en un solo click. Además, uso \lstinline!Jabref! para
gestionar la bibliografía y \lstinline!git! para tener todo bajo control
de versiones.

Cuando escribo en \textbf{Markdown para posteriormente compilar con
Pandoc} uso el modo Markdown de Emacs junto con unos atajos de teclado
que ejecutan \lstinline!make! que tengo definidos. Si añadimos además
una línea al Makefile para que nos abra directamente el
\emph{pdf}\footnote{\lstinline!xdg-open PDF! en escritorios que cumplan
  con Freedesktop y \lstinline!explorer.exe PDF &! en Windows.} tenemos
un entorno de edición que no tiene nada que envidiarle a ningún IDE.

\section{¿Qué paquetes uso?}

Aparte de los típicos paquetes de idioma (\lstinline!babel! o
\lstinline!polyglossia!) o de matemática (\lstinline!amsmath!,
\lstinline!amsthm!, \lstinline!amssymb!) de los que ya hemos hablado y
los básicos como \lstinline!xcolor! y \lstinline!graphicx!,
habitualmente utilizo los siguientes paquetes:

\begin{itemize}
\item
  \href{https://ctan.org/pkg/parskip}{\lstinline!parskip!} para separar
  los párrafos mediante una línea blanca en lugar de sangrarlos.
\item
  \href{https://www.ctan.org/pkg/listings}{\lstinline!listings!} para
  resaltar la sintaxis de los extractos de código.
\item
  \href{https://www.ctan.org/pkg/blindtext}{\lstinline!blindtext!} para
  generar documentos de prueba y testar el formato.
\item
  \href{https://ctan.org/pkg/hyperref}{\lstinline!hyperref!} para
  producir hipervínculos para las referencias cruzadas y bibliográficas
  así como incluir enlaces en el documento. Suele ser preferible
  cargarlo el último.
\item
  \href{https://ctan.org/pkg/fontspec}{\lstinline!fontspec!} para
  establecer la fuente del documento (para \lstinline!xelatex!).
\item
  \href{https://www.ctan.org/pkg/fancyhdr}{\lstinline!fancyhdr!} para
  personalizar los encabezados y pies de página.
\item
  \href{https://www.ctan.org/pkg/titlesec}{\lstinline!titlesec!} para
  modificar el estilo de los título de secciones y capítulos.
\item
  \href{https://ctan.org/tex-archive/macros/latex/contrib/booktabs/}{\lstinline!booktabs!}
  para tener más opciones para personalizar las tablas y que en general
  queden mejor.
\item
  \href{https://ctan.org/tex-archive/macros/latex/contrib/microtype/}{\lstinline!microtype!}
  para ajustar las opciones
  \href{https://en.wikipedia.org/wiki/Microtypography}{\emph{microtipográficas}}
  y que el documento tenga
  \href{http://www.khirevich.com/latex/microtype/Microtype_example_blurred_text_ani.gif}{mejor
  aspecto}.
\end{itemize}

\section{¿Cómo compilo?}

Suelo compilar con \lstinline!xelatex! porque, aunque es más pesado y
lento, me evita tener que configurar la codificación y es mucho más
sencillo cambiar de fuente. Si me temo que es posible que un documento
vaya a ser compilado con \lstinline!xelatex! y \lstinline!pdflatex! me
curo en salud, uso el paquete \lstinline!ifxetex! y defino las cosas
problemáticas, principalmente el idioma y la fuente, para ambos
compiladores:

\begin{lstlisting}[language={[latex]tex}]
\usepackage{ifxetex}

\ifxetex
  % Si se usa xelatex
  \usepackage{polyglossia}
  \setmainlanguage{spanish}
  
  % Fuente
  \usepackage{fontspec}
  \setmainfont{DejaVu Serif}
  
  % Tabla en lugar de cuadro
  \gappto\captionsspanish{
  \renewcommand{\tablename}{Tabla}%
  \renewcommand{\listtablename}{Índice de tablas}%
  }
  
\else
  % Si se usa pdflatex
  \usepackage[spanish,es-tabla]{babel}
  \usepackage[utf8]{inputenc} 
  \usepackage[T1]{fontenc}
  \usepackage{DejaVuSerif}
\fi
\end{lstlisting}

\section{¿Y para colaborar?}

Usando \lstinline!git!, evidentemente. Los que no sepan \lstinline!git!
que aprendan y los que no quieran aprender que se vayan a vivir a una
torre en el medio de la nada. El hecho de tener bajo control todos los
cambios, trabajar en paralelo sin pisarnos los unos a los otros y poder
probar cosas nuevas sin destruir lo que ya tenemos bien merece el
esfuerzo. Si además tenemos una copia de seguridad de nuestro trabajo en
un repositorio en la red y podemos comentar los cambios de los demás con
facilidad ni os cuento.

A los tengáis a alguien que os exige que cambiéis cosas y que luego se
lo demostréis os vendrá bien
\href{https://www.ctan.org/pkg/latexdiff?lang=en}{\lstinline!latexdiff!},
que genera un documento legible para los \emph{no iniciados} y no nos
cuesta un trabajo adicional.

\section{Conclusión final}

¡Hemos llegado al final! ¿Os dais cuenta de todo lo que sabemos ya?
Somos capaces de crear un documento profesional con sus referencias y su
código colorinesco que no haga que le sangren los ojos a las personas
con cierta \emph{educación tipográfica}. Y todo usando un par de
programas, unos paquetes exquisitamente escogidos y nuestras manitas.
Cómo molamos.

En definitiva, LaTeX no es tan fiero como lo pintan y cualquiera (¡hasta
yo!) puede aprender a usarlo si le dedica un poco de tiempo y ganas. Así
que ¡ahora mismo todos a generar documentos elegantes!

\section{Referencias}

\href{http://stackoverflow.com/questions/6188780/git-latex-workflow}{\emph{\lstinline!git!
+ LaTeX workflow} en StackOverflow}

\href{http://marianaeguaras.com/el-formato-de-una-publicacion-cuello-de-botella-en-la-edicion/}{\emph{El
formato de una publicación: cuello de botella en la edición}}

\href{http://www.khirevich.com/latex/microtype/}{\emph{Tips on Writing a
Thesis in LaTeX}}

\href{https://git-scm.com/book/en/v2}{\emph{Pro \lstinline!git!}}


\appendix
\chapter{Una nota sobre los archivos auxiliares}
\input{Contenido/Ap1.Auxiliares}

\chapter{Hablemos de paquetes}
\input{Contenido/Ap2.Paquetes}

\chapter{Enlaces de interés}
\input{Contenido/Ap3.Enlaces}


% Bibliografía
\backmatter
% \bibliographystyle{plain}
%\bibliography{bib}

\end{document}
