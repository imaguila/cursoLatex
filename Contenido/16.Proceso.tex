Hemos aprendido a crear documentos y presentaciones en LaTeX, sabemos
escribir macros propias y convertir archivos de un tipo a otro. Os
podría contar cómo cambiar la fuente del texto o hablar de diferentes
paquetes para modificar la apariencia de nuestro documento, pero creo
con lo que ya sabéis sois muy capaces de entender por vosotros mismos
cualquier paquete leyendo el manual. Así que en este último capítulo voy
a hablar de algo que desde mi punto de vista no se trata lo suficiente:
\textbf{cómo trabajo con LaTeX} en, por llamarlo de algún modo, un
entorno de producción. Veremos cómo organizo los archivos, qué software
y paquetes utilizo y algunas cosillas sobre compilación y colaboración.

\section{¿Cómo me organizo?}

Como he dicho montones de veces \emph{la organización es fundamental},
aunque no igual para todo el mundo. Cuando el documento es corto solo
separo las imágenes en una carpeta propia y escribo el contenido en el
mismo archivo en el que defino el estilo. Si, por el contrario, se trata
de un documento más extenso, como un libro, creo un archivo de LaTeX
principal desde el que llamo a las diferentes secciones o capítulos con
\lstinline!\include{}! o \lstinline!\input{}!.

A la hora de separar los archivos, tiendo a separarlos por tipo, lo que
me facilita aplicarles a todos ellos una misma acción\footnote{Pensad en
  cambiar el formato de todas las imágenes o buscar una palabra en el
  contenido.}. En ocasiones hago una segunda clasificación por capítulos
si tengo muchas figuras o extractos de código, por ejemplo.

\begin{lstlisting}
-- principal.tex
-- estilo.bst
-- referencias.bib

-- Contenido
.. -- 1.Intro.tex
.. -- 2.Segundo.tex
.. -- ...

-- Código
.. -- listing.py
.. -- ...

-- Figuras
    -- fig.eps
    -- ...
\end{lstlisting}

Si la definición del estilo es muy larga o estoy usando una clase que he
descargado por ahí, añado una carpeta extra para meter todo eso y no
volverme loca.

En cuanto al \textbf{software}, suelo escribir en Kile con el corrector
ortográfico activado, el ancho de línea fijado a 80 caracteres\footnote{Reducir
  el ancho de línea resulta útil a la hora de ver los cambios que hemos
  llevado a cabo en determinado archivo, \lstinline!git! nos pinta la
  línea modificada entera, si es gigante será más difícil encontrar la
  palabra exacta que hemos cambiado.} y una orden personalizada para
compilar que genera el documento con referencias cruzadas y
bibliográficas en un solo click. Además, uso \lstinline!Jabref! para
gestionar la bibliografía y \lstinline!git! para tener todo bajo control
de versiones.

Cuando escribo en \textbf{Markdown para posteriormente compilar con
Pandoc} uso el modo Markdown de Emacs junto con unos atajos de teclado
que ejecutan \lstinline!make! que tengo definidos. Si añadimos además
una línea al Makefile para que nos abra directamente el
\emph{pdf}\footnote{\lstinline!xdg-open PDF! en escritorios que cumplan
  con Freedesktop y \lstinline!explorer.exe PDF &! en Windows.} tenemos
un entorno de edición que no tiene nada que envidiarle a ningún IDE.

\section{¿Qué paquetes uso?}

Aparte de los típicos paquetes de idioma (\lstinline!babel! o
\lstinline!polyglossia!) o de matemática (\lstinline!amsmath!,
\lstinline!amsthm!, \lstinline!amssymb!) de los que ya hemos hablado y
los básicos como \lstinline!xcolor! y \lstinline!graphicx!,
habitualmente utilizo los siguientes paquetes:

\begin{itemize}
\item
  \href{https://ctan.org/pkg/parskip}{\lstinline!parskip!} para separar
  los párrafos mediante una línea blanca en lugar de sangrarlos.
\item
  \href{https://www.ctan.org/pkg/listings}{\lstinline!listings!} para
  resaltar la sintaxis de los extractos de código.
\item
  \href{https://www.ctan.org/pkg/blindtext}{\lstinline!blindtext!} para
  generar documentos de prueba y testar el formato.
\item
  \href{https://ctan.org/pkg/hyperref}{\lstinline!hyperref!} para
  producir hipervínculos para las referencias cruzadas y bibliográficas
  así como incluir enlaces en el documento. Suele ser preferible
  cargarlo el último.
\item
  \href{https://ctan.org/pkg/fontspec}{\lstinline!fontspec!} para
  establecer la fuente del documento (para \lstinline!xelatex!).
\item
  \href{https://www.ctan.org/pkg/fancyhdr}{\lstinline!fancyhdr!} para
  personalizar los encabezados y pies de página.
\item
  \href{https://www.ctan.org/pkg/titlesec}{\lstinline!titlesec!} para
  modificar el estilo de los título de secciones y capítulos.
\item
  \href{https://ctan.org/tex-archive/macros/latex/contrib/booktabs/}{\lstinline!booktabs!}
  para tener más opciones para personalizar las tablas y que en general
  queden mejor.
\item
  \href{https://ctan.org/tex-archive/macros/latex/contrib/microtype/}{\lstinline!microtype!}
  para ajustar las opciones
  \href{https://en.wikipedia.org/wiki/Microtypography}{\emph{microtipográficas}}
  y que el documento tenga
  \href{http://www.khirevich.com/latex/microtype/Microtype_example_blurred_text_ani.gif}{mejor
  aspecto}.
\end{itemize}

\section{¿Cómo compilo?}

Suelo compilar con \lstinline!xelatex! porque, aunque es más pesado y
lento, me evita tener que configurar la codificación y es mucho más
sencillo cambiar de fuente. Si me temo que es posible que un documento
vaya a ser compilado con \lstinline!xelatex! y \lstinline!pdflatex! me
curo en salud, uso el paquete \lstinline!ifxetex! y defino las cosas
problemáticas, principalmente el idioma y la fuente, para ambos
compiladores:

\begin{lstlisting}[language={[latex]tex}]
\usepackage{ifxetex}

\ifxetex
  % Si se usa xelatex
  \usepackage{polyglossia}
  \setmainlanguage{spanish}
  
  % Fuente
  \usepackage{fontspec}
  \setmainfont{DejaVu Serif}
  
  % Tabla en lugar de cuadro
  \gappto\captionsspanish{
  \renewcommand{\tablename}{Tabla}%
  \renewcommand{\listtablename}{Índice de tablas}%
  }
  
\else
  % Si se usa pdflatex
  \usepackage[spanish,es-tabla]{babel}
  \usepackage[utf8]{inputenc} 
  \usepackage[T1]{fontenc}
  \usepackage{DejaVuSerif}
\fi
\end{lstlisting}

\section{¿Y para colaborar?}

Usando \lstinline!git!, evidentemente. Los que no sepan \lstinline!git!
que aprendan y los que no quieran aprender que se vayan a vivir a una
torre en el medio de la nada. El hecho de tener bajo control todos los
cambios, trabajar en paralelo sin pisarnos los unos a los otros y poder
probar cosas nuevas sin destruir lo que ya tenemos bien merece el
esfuerzo. Si además tenemos una copia de seguridad de nuestro trabajo en
un repositorio en la red y podemos comentar los cambios de los demás con
facilidad ni os cuento.

A los tengáis a alguien que os exige que cambiéis cosas y que luego se
lo demostréis os vendrá bien
\href{https://www.ctan.org/pkg/latexdiff?lang=en}{\lstinline!latexdiff!},
que genera un documento legible para los \emph{no iniciados} y no nos
cuesta un trabajo adicional.

\section{Conclusión final}

¡Hemos llegado al final! ¿Os dais cuenta de todo lo que sabemos ya?
Somos capaces de crear un documento profesional con sus referencias y su
código colorinesco que no haga que le sangren los ojos a las personas
con cierta \emph{educación tipográfica}. Y todo usando un par de
programas, unos paquetes exquisitamente escogidos y nuestras manitas.
Cómo molamos.

En definitiva, LaTeX no es tan fiero como lo pintan y cualquiera (¡hasta
yo!) puede aprender a usarlo si le dedica un poco de tiempo y ganas. Así
que ¡ahora mismo todos a generar documentos elegantes!

\section{Referencias}

\href{http://stackoverflow.com/questions/6188780/git-latex-workflow}{\emph{\lstinline!git!
+ LaTeX workflow} en StackOverflow}

\href{http://marianaeguaras.com/el-formato-de-una-publicacion-cuello-de-botella-en-la-edicion/}{\emph{El
formato de una publicación: cuello de botella en la edición}}

\href{http://www.khirevich.com/latex/microtype/}{\emph{Tips on Writing a
Thesis in LaTeX}}

\href{https://git-scm.com/book/en/v2}{\emph{Pro \lstinline!git!}}
